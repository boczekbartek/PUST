%PROJ4
\chapter{Punkt 2}



\begin{figure}[ht]
	\centering
		
	\begin{tikzpicture}
	\begin{axis}[
	enlargelimits=false,
	width=4.667in,
	height=4in,
	%xmin=0,xmax=350
	ymin=-2.8,ymax=0.2,
	xlabel={$k$},
	ylabel={$Y$},
	legend pos=south east,
	y tick label style={/pgf/number format/1000 sep=},
	]
	\addplot[const plot,green] file {../wykresy/zad2odpowiedz_skokowa_Y_-1.txt};
	\addplot[const plot,yellow] file {../wykresy/zad2odpowiedz_skokowa_Y_-0.5.txt};
	\addplot[const plot,gray] file {../wykresy/zad2odpowiedz_skokowa_Y_0.txt};
	\addplot[const plot,red] file {../wykresy/zad2odpowiedz_skokowa_Y_0.5.txt};
	\addplot[const plot,blue] file {../wykresy/zad2odpowiedz_skokowa_Y_1.txt};
	\legend{$U=\num{-1}$,
			$U=\num{-0,5}$,
			$U=\num{0}$,
			$U=\num{0,5}$,
			$U=\num{1}$}

	\end{axis}
	\end{tikzpicture}
	
		\begin{tikzpicture}
	\begin{axis}[
	enlargelimits=false,
	width=4.667in,
	height=1.5in,
	%xmin=0,xmax=350
	%ymin=34,ymax=52,
	xlabel={$k$},
	ylabel={$Y$},
	legend pos=south east,
	y tick label style={/pgf/number format/1000 sep=},
	]
	\addplot[const plot,green] file {../wykresy/zad2odpowiedz_skokowa_U_-1.txt};
	\addplot[const plot,yellow] file {../wykresy/zad2odpowiedz_skokowa_U_-0.5.txt};
	\addplot[const plot,gray] file {../wykresy/zad2odpowiedz_skokowa_U_0.txt};
	\addplot[const plot,red] file {../wykresy/zad2odpowiedz_skokowa_U_0.5.txt};
	\addplot[const plot,blue] file {../wykresy/zad2odpowiedz_skokowa_U_1.txt};		
		\end{axis}
		\end{tikzpicture}
	
	\caption{Odpowiedzi skokowe procesu dla pi�ciu r�nych zmian sygna�u steruj�cego}
	
\end{figure}







\begin{figure}[ht]
	\centering
	
	\begin{tikzpicture}
	\begin{axis}[
	enlargelimits=false,
	width=4.667in,
	height=3in,
	%xmin=0,xmax=350
	%ymin=-2.8,ymax=0.2,
	xlabel={$U$},
	ylabel={$Y$},
	legend pos=south east,
	y tick label style={/pgf/number format/1000 sep=},
	]
	\addplot[const plot,green] file {../wykresy/char_stat.txt};
	\end{axis}
	\end{tikzpicture}
	
	\caption{Charakterystyka statyczna}
	
\end{figure}


\begin{figure}
	
	W�a�ciwo�ci statyczne obiektu nie s� w ca�o�ciowym uj�ciu- wzgl�dem dziedziny U - liniowe.
	Charakterystyka statyczna to z�o�enie dw�ch prostych, pierwszej szybko rosn�cej, drugiej bardzo wolno, niemal stale, z punktem przegi�cia w okolicach $\num{-0,1}$(u)
	W�a�ciwo�ci dynamiczne obiektu: uk�ad jest stabilny, ma op�nienie w ilo�ci 5 chwil k, ma jedn� lub dwie inercje; wszystkie te cechy dynamiczne mo�na uzna� za w przybli�eniu liniowe.
\end{figure}