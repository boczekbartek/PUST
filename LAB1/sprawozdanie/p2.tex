%LAB1
\chapter{Punkt 2}

Z punktu pracy $G1 = \num{28}\%$ wybralismy trzy skoki sygna�u sterujacego: do $\num{50}\%$, $\num{70}\%$ oraz
$\num{90}\%$.


\begin{figure}[ht]
	\centering
		
	\begin{tikzpicture}
	\begin{axis}[
	width=4.667in,
	height=5in,
	xmin=0,xmax=350,ymin=29,ymax=50,
	xlabel={$sek$},
	ylabel={$Y$},
	legend pos=south east,
	y tick label style={/pgf/number format/1000 sep=},
	]
	\addplot[const plot,green] file {wykresy/Y_50.txt};
	\addplot[const plot,red] file {wykresy/Y_70.txt};
	\addplot[const plot,blue] file {wykresy/Y_90.txt};
	\legend{$G1=\num{50}\%$,$G1=\num{70}\%$,$G1=\num{90}\%$}
	\end{axis}
	\end{tikzpicture}
	
	\caption{Odpowiedzi skokowe dla trzech r�nych zmian sygna�u steruj�cego}
	\label{odpWeWy}
\end{figure}



W�a�ciwo�ci statyczne obiektu mo�na okre�li� jako (w przybli�eniu) liniowe, bior�c pod uwag� du�e b��dy pomiaru zwi�zane ze zmian� temperatury (termometr zd��y� si� nagrza�). Je�eli zaproksymujemy charakterystyk� statyczn�, to z niewielkim b��dem b�dziemy mieli liniowe w�a�ciwo�ci statyczne. \\\\

$K_{\mathrm{statU_{\mathrm{pp}}->50}}$ = $\num{0,2586}$\\
$K_{\mathrm{stat50->70}}$ = $\num{0,2875}$\\
$K_{\mathrm{stat70->90}}$ = $\num{0,3155}$\\


