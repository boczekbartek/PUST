%LAB1
\chapter{Punkt 3}

W tym zadaniu szukali�my optymalnej aproksymacji otrzymanej przez nas odpowiedzi skokowej
dla skoku z wartosci $U_{\mathrm{pp}} = \num{28}$ do $U = \num{50}$. Podczas wyznaczania funkcji aproksymuj�cej
optymalizowali�my parametry $T1$, $T2$ oraz $K$ cz�onu inercyjnego drugiego rz�du z op�nieniem.
Optymalizacja polega�a na minimalizacji b��du �redniokwadratowego mi�dzy rzeczywist� odpowiedzi�
skokow�, a funkcj� aproksymujac�. Przy u�yciu optymalizatora $ga$ uda�o nam si� uzyska�
funkcj�, dla kt�rej b��d wyni�s� $E = \num{1,8765}$, przy warto�ciach parametr�w: $T1 = \num{17,099506}$;
$T2 = \num{64,970974}$; $K = \num{0,263644}$. Warto�� $T_d$ (op�znienia) eksperymentalnie ustawili�my na \num{10},
gdy� dla niej wyszed� najmniejszy b��d aproksymacji. Poni�ej przedstawione zosta�y wykresy
odpowiedzi skokowej oraz funkcji aproksymuj�cej.

\begin{figure}[ht]
	\centering
	\begin{tikzpicture}
	\begin{axis}[
	width=5.0in,
	height=3.7in,
	xmin=0,xmax=350,ymin=-0.05,ymax=0.3,
	xlabel={$k$},
	ylabel={$y$},
	legend pos=south east,
	y tick label style={/pgf/number format/1000 sep=},
	]
	\addplot[const plot,cyan] file {wykresy/odp_skokowa.txt};
	\addplot[red, densely dashed] file {wykresy/funkcja_aproksymujaca.txt};
	\end{axis}
	\end{tikzpicture}
	\caption{Odpowied� skokowa oryginalna i wersja aproksymowana}
	\label{odpskokUdmc3}
\end{figure}









