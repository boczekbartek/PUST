\documentclass[a4paper,titlepage,11pt,twosides,floatssmall]{mwrep}
\usepackage[left=2.5cm,right=2.5cm,top=2.5cm,bottom=2.5cm]{geometry}
\usepackage[OT1]{fontenc}
\usepackage{polski}
\usepackage{amsmath}
\usepackage{amsfonts}
\usepackage{amssymb}
\usepackage{graphicx}
\usepackage{url}
\usepackage{tikz}
\usetikzlibrary{arrows,calc,decorations.markings,math,arrows.meta}
\usepackage{rotating}
\usepackage[percent]{overpic}
\usepackage[cp1250]{inputenc}
\usepackage{xcolor}
\usepackage{pgfplots}
\usetikzlibrary{pgfplots.groupplots}
\usepackage{listings}
\usepackage{matlab-prettifier}
\usepackage{siunitx}
\usepackage{verbatim}
\definecolor{szary}{rgb}{0.95,0.95,0.95}
\sisetup{detect-weight,exponent-product=\cdot,output-decimal-marker={,},per-mode=symbol,binary-units=true,range-phrase={-},range-units=single}

%konfiguracje pakietu listings
\lstset{
	backgroundcolor=\color{szary},
	frame=single,
	breaklines=true,
}
\lstdefinestyle{customlatex}{
	basicstyle=\footnotesize\ttfamily,
	%basicstyle=\small\ttfamily,
}
\lstdefinestyle{customc}{
	breaklines=true,
	frame=tb,
	language=C,
	xleftmargin=0pt,
	showstringspaces=false,
	basicstyle=\small\ttfamily,
	keywordstyle=\bfseries\color{green!40!black},
	commentstyle=\itshape\color{purple!40!black},
	identifierstyle=\color{blue},
	stringstyle=\color{orange},
}
\lstdefinestyle{custommatlab}{
	captionpos=t,
	breaklines=true,
	frame=tb,
	xleftmargin=0pt,
	language=matlab,
	showstringspaces=false,
	%basicstyle=\footnotesize\ttfamily,
	basicstyle=\scriptsize\ttfamily,
	keywordstyle=\bfseries\color{green!40!black},
	commentstyle=\itshape\color{purple!40!black},
	identifierstyle=\color{blue},
	stringstyle=\color{orange},
}

%wymiar tekstu (bez �ywej paginy)
\textwidth 160mm \textheight 247mm

%ustawienia pakietu pgfplots
\pgfplotsset{
tick label style={font=\scriptsize},
label style={font=\small},
legend style={font=\small},
title style={font=\small}
}

\def\figurename{Rys.}
\def\tablename{Tab.}

%konfiguracja liczby p�ywaj�cych element�w
\setcounter{topnumber}{0}%2
\setcounter{bottomnumber}{3}%1
\setcounter{totalnumber}{5}%3
\renewcommand{\textfraction}{0.01}%0.2
\renewcommand{\topfraction}{0.95}%0.7
\renewcommand{\bottomfraction}{0.95}%0.3
\renewcommand{\floatpagefraction}{0.35}%0.5

\begin{document}
\frenchspacing
\pagestyle{uheadings}

%strona tytu�owa
\title{\bf Sprawozdanie z projektu i �wiczenia laboratoryjnego nr 1, zadanie nr 3\vskip 0.1cm}
\author{Bart�omiej Boczek, Aleksander Piotrowski, �ukasz �migielski}
\date{2017}

\makeatletter
\renewcommand{\maketitle}{\begin{titlepage}
\begin{center}{\LARGE {\bf
Wydzia� Elektroniki i Technik Informacyjnych}}\\
\vspace{0.4cm}
{\LARGE {\bf Politechnika Warszawska}}\\
\vspace{0.3cm}
\end{center}
\vspace{5cm}
\begin{center}
{\bf \LARGE Projektowanie uk�ad�w sterowania\\ (projekt grupowy) \vskip 0.1cm}
\end{center}
\vspace{1cm}
\begin{center}
{\bf \LARGE \@title}
\end{center}
\vspace{2cm}
\begin{center}
{\bf \Large \@author \par}
\end{center}
\vspace*{\stretch{6}}
\begin{center}
\bf{\large{Warszawa, \@date\vskip 0.1cm}}
\end{center}
\end{titlepage}
}
\makeatother

\maketitle

\tableofcontents
\chapter{Punkt 1}
Wyjscie procesu zbiega do $Y_{\mathrm{pp}}$ przy sygnale sterujacym rownym  $U_{\mathrm{pp}}$. Potwierdza to poprawnosc wartosci punktu pracy.

\begin{figure}[h]
	\centering
	\begin{tikzpicture}
	\begin{axis}[
	width=0.5\textwidth,
	xmin=1,xmax=300,ymin=1.5,ymax=2.5,
	xlabel={$k$},
	ylabel={$y$},
	xtick={1,100, 200, 300},
	ytick={1.5, 2, 2.5},
	legend pos=south east,
	y tick label style={/pgf/number format/1000 sep=},
	]
	\addplot[red,semithick] file {wykresy/zadanie1Y.txt};

	\end{axis}
	\end{tikzpicture}
	\caption{Wartosci sygnalu wyjsciowego przy stalym sygnale wejsciowym $U = U_{\mathrm{pp}} = \num{1,1}$}
	\label{r_pgfplots_funkcje}
\end{figure}

\begin{figure}[h]
	\centering
	\begin{tikzpicture}
	\begin{axis}[
	width=0.5\textwidth,
	xmin=1,xmax=300,ymin=0.5,ymax=1.5,
	xlabel={$k$},
	ylabel={$u$},
	xtick={1,100, 200, 300},
	ytick={0.5, 1, 1.5},
	legend pos=south east,
	y tick label style={/pgf/number format/1000 sep=},
	]
	\addplot[red,semithick] file {wykresy/zadanie1U.txt};

	\end{axis}
	\end{tikzpicture}
	\caption{Sygnal wejsciowy}
	\label{r_pgfplots_funkcje}
\end{figure}


\chapter{Punkt 2}

Wyznaczylismy symulacyjnie odpowiedzi skokowe dla czterech zmian sygnalu sterujacego od $U_{\mathrm{pp}}$ do, kolejno, $\num{0,9}$,  $\num{1,0}$,  $\num{1,2}$,  $\num{1,3}$.

%\begin{comment}
\begin{figure}[h]
	\centering
	\begin{tikzpicture}
	\begin{axis}[
	width=0.5\textwidth,
	xmin=1,xmax=150,ymin=0,ymax=3,
	xlabel={$k$},
	ylabel={$y$},
	xtick={1,50,100,150},
	ytick={0,1,2,3},
	legend pos=south east,
	y tick label style={/pgf/number format/1000 sep=},
	]
	\addplot[red,semithick] file {wykresy/zadanie2_jump=1.3.txt};
	\addplot[blue,semithick] file {wykresy/zadanie2_jump=1.2.txt};
	\addplot[green,semithick] file {wykresy/zadanie2_jump=1.txt};
	\addplot[black,semithick] file {wykresy/zadanie2_jump=0.9.txt};
	\legend{$U=\num{1,3}$, $U=\num{1,2}$, $U=\num{1,0}$, $U=\num{0,9}$}
	\end{axis}
	\end{tikzpicture}
	\caption{Odpowiedzi skokowe dla czterech zmian sygnalu sterujacego}
	\label{r_pgfplots_funkcje}
\end{figure}
%\end{comment}

\begin{comment}
\begin{figure}[h]
	\centering
	\begin{tikzpicture}
	\begin{axis}[
	width=0.5\textwidth,
	xmin=1,xmax=300,ymin=0,ymax=2,
	xlabel={$k$},
	ylabel={$y$},
	xtick={1,100, 200, 200},
	ytick={0, 1, 2},
	legend pos=south east,
	y tick label style={/pgf/number format/1000 sep=},
	]
	\addplot[red,semithick] file {JAKAS_NAZWA.txt};
	\legend{$y$}
	\end{axis}
	\end{tikzpicture}
	\caption{Charakterystyka statyczna}
	\label{r_pgfplots_funkcje}
\end{figure}
\end{comment}
//TODO
Czy wlasciwosci statyczne i dynamiczne procesu sa w przyblizeniu liniowe? Jesli tak, okreslic wzmocnienie statyczne procesu





\chapter{Punkt 3}
W~praktyce bardzo cz�sto nale�y wyr�wna� liczby wzgl�dem cyfr znacz�cych w~poszczeg�lnych kolumnach (czyli przecinek dziesi�tny ma by� we wszystkich 


\chapter{Punkt 4}
Wszystkie elementy dokumentu opracowanego w~systemie \LaTeX powinny wygl�da� jednolicie. Do wykonywania rysunk�w korzystamy wi�c z mechanizm�w 



















\end{document}


