\chapter{Punkt 1}
Wyjscie procesu zbiega do $Y_{\mathrm{pp}}$ przy sygnale sterujacym rownym  $U_{\mathrm{pp}}$. Potwierdza to poprawnosc wartosci punktu pracy.

\begin{figure}[h]
	\centering
	\begin{tikzpicture}
	\begin{axis}[
	width=0.5\textwidth,
	xmin=1,xmax=300,ymin=1.5,ymax=2.5,
	xlabel={$k$},
	ylabel={$y$},
	xtick={1,100, 200, 300},
	ytick={1.5, 2, 2.5},
	legend pos=south east,
	y tick label style={/pgf/number format/1000 sep=},
	]
	\addplot[red,semithick] file {wykresy/zadanie1Y.txt};

	\end{axis}
	\end{tikzpicture}
	\caption{Wartosci sygnalu wyjsciowego przy stalym sygnale wejsciowym $U = U_{\mathrm{pp}} = \num{1,1}$}
	\label{r_pgfplots_funkcje}
\end{figure}

\begin{figure}[h]
	\centering
	\begin{tikzpicture}
	\begin{axis}[
	width=0.5\textwidth,
	xmin=1,xmax=300,ymin=0.5,ymax=1.5,
	xlabel={$k$},
	ylabel={$u$},
	xtick={1,100, 200, 300},
	ytick={0.5, 1, 1.5},
	legend pos=south east,
	y tick label style={/pgf/number format/1000 sep=},
	]
	\addplot[red,semithick] file {wykresy/zadanie1U.txt};

	\end{axis}
	\end{tikzpicture}
	\caption{Sygnal wejsciowy}
	\label{r_pgfplots_funkcje}
\end{figure}


\chapter{Punkt 2}

Wyznaczylismy symulacyjnie odpowiedzi skokowe dla czterech zmian sygnalu sterujacego od $U_{\mathrm{pp}}$ do, kolejno, $\num{0,9}$,  $\num{1,0}$,  $\num{1,2}$,  $\num{1,3}$.

%\begin{comment}
\begin{figure}[h]
	\centering
	\begin{tikzpicture}
	\begin{axis}[
	width=0.5\textwidth,
	xmin=1,xmax=150,ymin=0,ymax=3,
	xlabel={$k$},
	ylabel={$y$},
	xtick={1,50,100,150},
	ytick={0,1,2,3},
	legend pos=south east,
	y tick label style={/pgf/number format/1000 sep=},
	]
	\addplot[red,semithick] file {wykresy/zadanie2_jump=1.3.txt};
	\addplot[blue,semithick] file {wykresy/zadanie2_jump=1.2.txt};
	\addplot[green,semithick] file {wykresy/zadanie2_jump=1.txt};
	\addplot[black,semithick] file {wykresy/zadanie2_jump=0.9.txt};
	\legend{$U=\num{1,3}$, $U=\num{1,2}$, $U=\num{1,0}$, $U=\num{0,9}$}
	\end{axis}
	\end{tikzpicture}
	\caption{Odpowiedzi skokowe dla czterech zmian sygnalu sterujacego}
	\label{r_pgfplots_funkcje}
\end{figure}
%\end{comment}

\begin{comment}
\begin{figure}[h]
	\begin{tikzpicture}

	%%1
	[xmin=1.1,xmax=1.3,ymin=2,ymax=3,
	xtick={1.1, 1.2, 1.3},ytick={2, 2.5, 3},
	xlabel={$u$},ylabel={$y$},legend cell align=left,legend pos=north east]
	\addplot[const plot,color=blue,semithick] file {wykresy/zadanie2_wzmStat.txt};

	\end{tikzpicture}
\end{figure}
\end{comment}


\begin{figure}[h]
	\centering
	\begin{tikzpicture}
	\begin{axis}[
	width=0.5\textwidth,
	xmin=1.1,xmax=1.3,ymin=2,ymax=3,
	xlabel={$u$},
	ylabel={$y$},
	xtick={1.1, 1.2, 1.3},
	ytick={2, 2.5, 3},
	legend pos=south east,
	y tick label style={/pgf/number format/1000 sep=},
	]
	\addplot[red,semithick] file {wykresy/zadanie2_wzmStat.txt};

%	\legend{$U=\num{1,3}$, $U=\num{1,2}$, $U=\num{1,0}$, $U=\num{0,9}$}
	\end{axis}
	\end{tikzpicture}
	\caption{Wzmocnienie statyczne}
	\label{r_pgfplots_funkcje}
\end{figure}


//TODO
Czy wlasciwosci statyczne i dynamiczne procesu sa w przyblizeniu liniowe? Jesli tak, okreslic wzmocnienie statyczne procesu





\chapter{Punkt 3}
 
\begin{figure}[h]
	\centering
	\begin{tikzpicture}
	\begin{axis}[
	width=0.5\textwidth,
	xmin=1,xmax=130,ymin=0,ymax=3.5,
	xlabel={$k$},
	ylabel={$y$},
	xtick={1, 50, 100, 130},
	ytick={0, 1, 2, 3, 3.5},
	legend pos=south east,
	y tick label style={/pgf/number format/1000 sep=},
	]
	\addplot[const plot,color=blue,semithick] file {wykresy/zadanie3_odpSkok_schodki.txt};
	
	%	\legend{$U=\num{1,3}$, $U=\num{1,2}$, $U=\num{1,0}$, $U=\num{0,9}$}
	\end{axis}
	\end{tikzpicture}
	\caption{Odpowiedz skokowa?????????}
	\label{r_pgfplots_funkcje}
\end{figure}	



\chapter{Punkt 4}
Wszystkie elementy dokumentu opracowanego w~systemie \LaTeX powinny wygl�da� jednolicie. Do wykonywania rysunk�w korzystamy wi�c z mechanizm�w 


















