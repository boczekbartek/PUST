%LAB3
\chapter{Punkt 3}
%s^{i,j} - ite wyj�cie pod wp�ywem jotego wej�cia

Uzasadni� wyb�r parametr�w optymalizacji.

\begin{figure}[ht]
	\centering
	\begin{tikzpicture}
	\begin{axis}[
	width=5.0in,
	height=3.4in,
	xmin=0,xmax=320,ymin=-0.1,ymax=0.2,
	xlabel={$k$},
	ylabel={$s^{\mathrm{1,1}}$},
	legend pos=south east,
	y tick label style={/pgf/number format/1000 sep=},
	]
	\addplot[const plot,cyan] file {wykresy/przykladowe/zadanie3_sz.txt};
	\end{axis}
	\end{tikzpicture}
	\caption{Odpowied� skokowa wyj�cia 1 przy skoku wej�cia 1}
	\label{odpskokUdmc3}
\end{figure}

\begin{figure}[ht]
	\centering
	\begin{tikzpicture}
	\begin{axis}[
	width=5.0in,
	height=3.4in,
	xmin=0,xmax=320,ymin=-0.1,ymax=0.2,
	xlabel={$k$},
	ylabel={$s^{\mathrm{1,2}}$},
	legend pos=south east,
	y tick label style={/pgf/number format/1000 sep=},
	]
	\addplot[const plot,cyan] file {wykresy/przykladowe/zadanie3_sz.txt};
	\end{axis}
	\end{tikzpicture}
	\caption{Odpowied� skokowa wyj�cia 1 przy skoku wej�cia 2}
	\label{odpskokUdmc3}
\end{figure}

\begin{figure}[ht]
	\centering
	\begin{tikzpicture}
	\begin{axis}[
	width=5.0in,
	height=3.4in,
	xmin=0,xmax=320,ymin=-0.1,ymax=0.2,
	xlabel={$k$},
	ylabel={$s^{\mathrm{2,1}}$},
	legend pos=south east,
	y tick label style={/pgf/number format/1000 sep=},
	]
	\addplot[const plot,cyan] file {wykresy/przykladowe/zadanie3_sz.txt};
	\end{axis}
	\end{tikzpicture}
	\caption{Odpowied� skokowa wyj�cia 2 przy skoku wej�cia 1}
	\label{odpskokUdmc3}
\end{figure}


\begin{figure}[ht]
	\centering
	\begin{tikzpicture}
	\begin{axis}[
	width=5.0in,
	height=3.4in,
	xmin=0,xmax=320,ymin=-0.1,ymax=0.2,
	xlabel={$k$},
	ylabel={$s^{\mathrm{2,2}}$},
	legend pos=south east,
	y tick label style={/pgf/number format/1000 sep=},
	]
	\addplot[const plot,cyan] file {wykresy/przykladowe/zadanie3_sz.txt};
	\end{axis}
	\end{tikzpicture}
	\caption{Odpowied� skokowa wyj�cia 2 przy skoku wej�cia 2}
	\label{odpskokUdmc3}
\end{figure}

\begin{figure}[ht]
	\centering
	\begin{tikzpicture}
	\begin{axis}[
	width=5.0in,
	height=3.4in,
	xmin=0,xmax=320,ymin=-0.01,ymax=0.4,
	xlabel={$k$},
	ylabel={$Y$},
	legend pos=south east,
	y tick label style={/pgf/number format/1000 sep=},
	]
	\addplot[const plot,red] file {wykresy/przykladowe/zadanie3_s.txt};
	\addplot[cyan] file {wykresy/przykladowe/zadanie3_saproks.txt};
	\legend{orygina�,aproksymacja}
	\end{axis}
	\end{tikzpicture}
	\caption{Por�wnanie odpowiedzi skokowej (wyj�cia 1 przy skoku wej�cia 1) i aproksymowanej}
	\label{odpskokUdmc4}
\end{figure}

\begin{figure}[ht]
	\centering
	\begin{tikzpicture}
	\begin{axis}[
	width=5.0in,
	height=3.4in,
	xmin=0,xmax=320,ymin=-0.01,ymax=0.4,
	xlabel={$k$},
	ylabel={$Y$},
	legend pos=south east,
	y tick label style={/pgf/number format/1000 sep=},
	]
	\addplot[const plot,red] file {wykresy/przykladowe/zadanie3_s.txt};
	\addplot[cyan] file {wykresy/przykladowe/zadanie3_saproks.txt};
	\legend{orygina�,aproksymacja}
	\end{axis}
	\end{tikzpicture}
	\caption{Por�wnanie odpowiedzi skokowej (wyj�cia 1 przy skoku wej�cia 2) i aproksymowanej}
	\label{odpskokUdmc4}
\end{figure}

\begin{figure}[ht]
	\centering
	\begin{tikzpicture}
	\begin{axis}[
	width=5.0in,
	height=3.4in,
	xmin=0,xmax=320,ymin=-0.01,ymax=0.4,
	xlabel={$k$},
	ylabel={$Y$},
	legend pos=south east,
	y tick label style={/pgf/number format/1000 sep=},
	]
	\addplot[const plot,red] file {wykresy/przykladowe/zadanie3_s.txt};
	\addplot[cyan] file {wykresy/przykladowe/zadanie3_saproks.txt};
	\legend{orygina�,aproksymacja}
	\end{axis}
	\end{tikzpicture}
	\caption{Por�wnanie odpowiedzi skokowej (wyj�cia 2 przy skoku wej�cia 1) i aproksymowanej}
	\label{odpskokUdmc4}
\end{figure}

\begin{figure}[ht]
	\centering
	\begin{tikzpicture}
	\begin{axis}[
	width=5.0in,
	height=3.4in,
	xmin=0,xmax=320,ymin=-0.01,ymax=0.4,
	xlabel={$k$},
	ylabel={$Y$},
	legend pos=south east,
	y tick label style={/pgf/number format/1000 sep=},
	]
	\addplot[const plot,red] file {wykresy/przykladowe/zadanie3_s.txt};
	\addplot[cyan] file {wykresy/przykladowe/zadanie3_saproks.txt};
	\legend{orygina�,aproksymacja}
	\end{axis}
	\end{tikzpicture}
	\caption{Por�wnanie odpowiedzi skokowej (wyj�cia 2 przy skoku wej�cia 2) i aproksymowanej}
	\label{odpskokUdmc4}
\end{figure}