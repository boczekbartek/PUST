%LAB3
\chapter{Punkt 1 }


Uda�o nam si� skomunikowa� sterownik z obiektem grzej�co-ch�odz�cym oraz oprogramowanie \verb|GxWorks3|. Opis rejestr�w sterownika:\\\\

Grza�ki:\\
D100 (temperatura T1: sygna� Y1) - pomiar temperatury na grza�ce G1 (np warto�� 2300 oznacza $\num{23}\degree$C)\\
D102 (temperatura T3: sygna� Y2) - pomiar temperatury na grza�ce G3\\\\

D114 (sygna� U1) - sygna� steruj�cy grza�k� G1 (zakres 0-1000 czyli 0-100\%)\\
D115 (sygna� U2) - sygna� steruj�cy grza�k� G2 \\\\

Wiatraczki:\\
D110 - sygna� steruj�cy wiatraczka W1   (zakres 0-1000 czyli 0-100\%)\\
D111 - sygna� steruj�cy wiatraczka W2\\\\

Aby ustali� punkt pracy wyprowadzili�my obiekt z r�wnowagi a nast�pnie podali�my mu sta�e sterowanie na obie grza�ki, przy wiatraczkach ustawionych na 50\% mocy.\\
Warto�ci sterowania:\\
G1 = 35 \%\\
G2 = 40 \%\\\\

Poni�ej przedstawiony jest przebieg eksperymentu:



\begin{figure}[ht]
	\centering
	\begin{tikzpicture}
	\begin{axis}[
	enlargelimits=false,
	width=5.0in,
	height=2.0in,
	%xmin=0,xmax=300,
	ymin=3500,ymax=8000,
	%xlabel={$k$},
	ylabel={$Y1$},
	]
	\addplot[const plot,cyan] file {wykresy/zadanie1_Y1.txt};
	\end{axis}
	\end{tikzpicture}

	\begin{tikzpicture}
	\begin{axis}[
	enlargelimits=false,
	width=5.0in,
	height=2.0in,
	%xmin=0,xmax=300,
	ymin=3500,ymax=8000,
	%xlabel={$k$},
	ylabel={$Y2$},
	]
	\addplot[const plot,cyan] file {wykresy/zadanie1_Y2.txt};
	\end{axis}
	\end{tikzpicture}
	
	\begin{tikzpicture}
	\begin{axis}[
	enlargelimits=false,
	width=5.0in,
	height=1.1in,
	%xmin=0,xmax=300,ymin=32,ymax=34,
	%xlabel={$k$},
	ylabel={$U1$},
	]
	\addplot[const plot,cyan] file {wykresy/zadanie1_U1.txt};
	\end{axis}
	\end{tikzpicture}
	
	\begin{tikzpicture}
	\begin{axis}[
	enlargelimits=false,
	width=5.0in,
	height=1.1in,
	%xmin=0,xmax=300,ymin=32,ymax=34,
	xlabel={$k$},
	ylabel={$U2$},
	]
	\addplot[const plot,cyan] file {wykresy/zadanie1_U2.txt};
	\end{axis}
	\end{tikzpicture}

	\caption{Sygna�y wyj�ciowe i wej�ciowe procesu}
	
\end{figure}




\begin{figure}
Jak wida� warto�ci temperatur w punkcie pracy prezentuj� si� nast�puj�co:\\
Y1 = $\num{42,5}\degree$C\\ 
Y2 =  $\num{39}\degree$C\\
	
\end{figure}