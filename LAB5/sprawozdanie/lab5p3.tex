%LAB4
\chapter{Punkt 3}

Nastawy PID:\\
(takie jak na labolatorium 3) 	\\
$K1=\num{1,228038}; T_i1=\num{11,602531}; T_d1=\num{1,166250}; $\\
$K2=\num{0,851927}; T_i2=\num{9,777445}; T_d2=\num{0,842038};$\\\\

W algorytmie zosta�y u�yte wcze�niej wyliczone w Matlabie paramtry RR12, RR10, RR10
RR22, RR21, RR20, tak aby zwolni� sterownik z oblicze� i pro�ci� implementacj�.\\\\

Implementacja na sterowniku wraz z zabezpieczeniem opisanym w zadaniu 2.

\begin{lstlisting}
AWARIAY1 := FALSE;
AWARIAY2 := FALSE;
YZAD1 := 4000;
YZAD2 := 4000;

YY1 := INT_TO_REAL(D100);
YY2 := INT_TO_REAL(D102);

IF (YY1 >15000.0) THEN
	AWARIAY1 := TRUE;
END_IF;

IF (YY2 > 15000.0) THEN
	AWARIAY2 := TRUE;
END_IF;

IF (NOT AWARIAY1 AND NOT AWARIAY2) THEN
	YY1 := YY1-Y1pp;
	YY2 := YY2-Y2pp;
	
	EE1 := YZAD1-YY1;
	EE2 := YZAD2-YY2;
	
	du1 := (RR12*e1_2)+(RR11*e1_1)+(RR10*EE1);
	du2 := (RR22*e2_2)+(RR21*e2_1)+(RR20*EE2);
	
	
	UU1 := UU1+du1+U1pp;
	UU2 := UU2+du2+U2pp;
	
	e1_2 := e1_1;
	e2_2 := e2_1;
	
	e1_1 := EE1;
	e2_1 := EE2;
	
	
	D110 := 500;
	D111 := 500;
	
	IF (UU1>1000.0) THEN
		UU1 := 1000.0;
	END_IF;
	
	IF (UU1<0.0) THEN
		UU1 := 0.0;
	END_IF;

END_IF;


IF (UU2>1000.0) THEN
	UU2 := 1000.0;
END_IF;

IF (UU2<0.0) THEN
	UU2 := 0.0;
END_IF;


//ZACHOWANIE W PRZYPADKU AWARII
IF(AWARIAY1) THEN
	UU1 := 0.0;
END_IF;

IF(AWARIAY2) THEN
	UU2 := 0.0;
END_IF;


D114 := REAL_TO_INT(UU1);
D115 := REAL_TO_INT(UU2);
\end{lstlisting}

\begin{figure}
	Na poni�szych wykresach mo�emy zaobserwowa� dzia�anie regulatora dla warto�ci zadanej
	dla obu temperatur, T1 oraz T3 = $\num{50}\degree$C. Pocz�tkowy stan wynosi� ok. T1 = T3 = $\num{25}\degree$C.
\end{figure}


\begin{figure}[ht]
	\centering
	\begin{tikzpicture}
	\begin{axis}[
	enlargelimits=false,
	width=5.0in,
	height=2.5in,
	%xmin=0,xmax=300,
	ymin=2000,ymax=6000,
	%xlabel={$k$},
	ylabel={$Y1$},
	legend pos=south east,
	]
	\addplot[const plot,cyan] file {wykresy/zadanie3_Y1.txt};
	\addplot[red] file {wykresy/zadanie3_Y1zad.txt};
	\legend{$Y1$, $Y1_{zad}$}
	\end{axis}
	\end{tikzpicture}
	
	\begin{tikzpicture}
	\begin{axis}[
	enlargelimits=false,
	width=5.0in,
	height=2.5in,
	%xmin=0,xmax=300,
	ymin=2000,ymax=6000,
	%xlabel={$k$},
	ylabel={$Y2$},
	legend pos=south east,
	]
	\addplot[const plot,cyan] file {wykresy/zadanie3_Y2.txt};
	\addplot[red] file {wykresy/zadanie3_Y2zad.txt};
	\legend{$Y2$, $Y2_{zad}$}
	\end{axis}
	\end{tikzpicture}
	
	\begin{tikzpicture}
	\begin{axis}[
	enlargelimits=false,
	width=5.0in,
	height=1.9in,
	%xmin=0,xmax=300,
	ymin=-100,ymax=1100,
	%xlabel={$k$},
	ylabel={$U1$},
	]
	\addplot[const plot,cyan] file {wykresy/zadanie3_U1.txt};
	\end{axis}
	\end{tikzpicture}
	
	\begin{tikzpicture}
	\begin{axis}[
	enlargelimits=false,
	width=5.0in,
	height=1.9in,
	%xmin=0,xmax=300,
	ymin=-100,ymax=1100,
	xlabel={$k$},
	ylabel={$U2$},
	]
	\addplot[const plot,cyan] file {wykresy/zadanie3_U2.txt};
	\end{axis}
	\end{tikzpicture}
	
	\caption{Sygna�y wyj�ciowe i wej�ciowe procesu}
	
\end{figure}
