%LAB4
\documentclass[a4paper,titlepage,11pt,twosides,floatssmall]{mwrep}
\usepackage[left=2.5cm,right=2.5cm,top=2.5cm,bottom=2.5cm]{geometry}
\usepackage[OT1]{fontenc}
\usepackage{polski}
\usepackage{amsmath}
\usepackage{amsfonts}
\usepackage{amssymb}
\usepackage{graphicx}
\usepackage{url}
\usepackage{tikz}
\usetikzlibrary{arrows,calc,decorations.markings,math,arrows.meta}
\usepackage{rotating}
\usepackage[percent]{overpic}
\usepackage[cp1250]{inputenc}
\usepackage{xcolor}
\usepackage{pgfplots}
\usetikzlibrary{pgfplots.groupplots}
\usepackage{listings}
\usepackage{matlab-prettifier}
\usepackage{siunitx}
\usepackage{placeins}
\usepackage{gensymb}
\usepackage{verbatim}
\definecolor{szary}{rgb}{0.95,0.95,0.95}
\sisetup{detect-weight,exponent-product=\cdot,output-decimal-marker={,},per-mode=symbol,binary-units=true,range-phrase={-},range-units=single}
\SendSettingsToPgf

%konfiguracje pakietu listings
\lstset{
	backgroundcolor=\color{szary},
	frame=single,
	breaklines=true,
}
\lstdefinestyle{customlatex}{
	basicstyle=\footnotesize\ttfamily,
	%basicstyle=\small\ttfamily,
}
\lstdefinestyle{customc}{
	breaklines=true,
	frame=tb,
	language=C,
	xleftmargin=0pt,
	showstringspaces=false,
	basicstyle=\small\ttfamily,
	keywordstyle=\bfseries\color{green!40!black},
	commentstyle=\itshape\color{purple!40!black},
	identifierstyle=\color{blue},
	stringstyle=\color{orange},
}
\lstdefinestyle{custommatlab}{
	captionpos=t,
	breaklines=true,
	frame=tb,
	xleftmargin=0pt,
	language=matlab,
	showstringspaces=false,
	%basicstyle=\footnotesize\ttfamily,
	basicstyle=\scriptsize\ttfamily,
	keywordstyle=\bfseries\color{green!40!black},
	commentstyle=\itshape\color{purple!40!black},
	identifierstyle=\color{blue},
	stringstyle=\color{orange},
}

%wymiar tekstu (bez �ywej paginy)
\textwidth 160mm \textheight 247mm

%ustawienia pakietu pgfplots
\pgfplotsset{
	tick label style={font=\scriptsize},
	label style={font=\small},
	legend style={font=\small},
	title style={font=\small}
}

\def\figurename{Rys.}
\def\tablename{Tab.}

%konfiguracja liczby p�ywaj�cych element�w
\setcounter{topnumber}{0}%2
\setcounter{bottomnumber}{3}%1
\setcounter{totalnumber}{5}%3
\renewcommand{\textfraction}{0.01}%0.2
\renewcommand{\topfraction}{0.95}%0.7
\renewcommand{\bottomfraction}{0.95}%0.3
\renewcommand{\floatpagefraction}{0.35}%0.5

\begin{document}
	\frenchspacing
	\pagestyle{uheadings}
	
	%strona tytu�owa
	\title{\bf Sprawozdanie z �wiczenia laboratoryjnego nr 5\vskip 0.1cm}
	\author{Bart�omiej Boczek, Aleksander Piotrowski, �ukasz �migielski}
	\date{11 czerwca 2017}
	
	\makeatletter
	\renewcommand{\maketitle}{\begin{titlepage}
			\begin{center}{\LARGE {\bf
						Wydzia� Elektroniki i Technik Informacyjnych}}\\
				\vspace{0.4cm}
				{\LARGE {\bf Politechnika Warszawska}}\\
				\vspace{0.3cm}
			\end{center}
			\vspace{5cm}
			\begin{center}
				{\bf \LARGE Projektowanie uk�ad�w sterowania\\ (projekt grupowy) \vskip 0.1cm}
			\end{center}
			\vspace{1cm}
			\begin{center}
				{\bf \LARGE \@title}
			\end{center}
			\vspace{2cm}
			\begin{center}
				{\bf \Large \@author \par}
			\end{center}
			\vspace*{\stretch{6}}
			\begin{center}
				\bf{\large{Warszawa, \@date\vskip 0.1cm}}
			\end{center}
		\end{titlepage}
	}
	\makeatother
	
	\maketitle
	
	\tableofcontents

	%LAB3
\chapter{Punkt 1 }


Uda�o nam si� skomunikowa� sterownik z obiektem grzej�co-ch�odz�cym oraz oprogramowanie \verb|GxWorks3|. Opis rejestr�w sterownika:\\\\

Grza�ki:\\
D100 (temperatura T1: sygna� Y1) - pomiar temperatury na grza�ce G1 (np warto�� 2300 oznacza $\num{23}\degree$C)\\
D102 (temperatura T3: sygna� Y2) - pomiar temperatury na grza�ce G3\\\\

D114 (sygna� U1) - sygna� steruj�cy grza�k� G1 (zakres 0-1000 czyli 0-100\%)\\
D115 (sygna� U2) - sygna� steruj�cy grza�k� G2 \\\\

Wiatraczki:\\
D110 - sygna� steruj�cy wiatraczka W1   (zakres 0-1000 czyli 0-100\%)\\
D111 - sygna� steruj�cy wiatraczka W2\\\\

Aby ustali� punkt pracy wyprowadzili�my obiekt z r�wnowagi a nast�pnie podali�my mu sta�e sterowanie na obie grza�ki, przy wiatraczkach ustawionych na 50\% mocy.\\
Warto�ci sterowania:\\
G1 = 35 \%\\
G2 = 40 \%\\\\
 Czas pr�bkowania zosta� ustawiony na 4 sekundy.\\\\
Poni�ej przedstawiony jest przebieg eksperymentu:



\begin{figure}[ht]
	\centering
	\begin{tikzpicture}
	\begin{axis}[
	enlargelimits=false,
	width=5.0in,
	height=2.0in,
	%xmin=0,xmax=300,
	ymin=3500,ymax=8000,
	%xlabel={$k$},
	ylabel={$Y1$},
	]
	\addplot[const plot,cyan] file {wykresy/zadanie1_Y1.txt};
	\end{axis}
	\end{tikzpicture}

	\begin{tikzpicture}
	\begin{axis}[
	enlargelimits=false,
	width=5.0in,
	height=2.0in,
	%xmin=0,xmax=300,
	ymin=3500,ymax=8000,
	%xlabel={$k$},
	ylabel={$Y2$},
	]
	\addplot[const plot,cyan] file {wykresy/zadanie1_Y2.txt};
	\end{axis}
	\end{tikzpicture}
	
	\begin{tikzpicture}
	\begin{axis}[
	enlargelimits=false,
	width=5.0in,
	height=1.1in,
	%xmin=0,xmax=300,ymin=32,ymax=34,
	%xlabel={$k$},
	ylabel={$U1$},
	]
	\addplot[const plot,cyan] file {wykresy/zadanie1_U1.txt};
	\end{axis}
	\end{tikzpicture}
	
	\begin{tikzpicture}
	\begin{axis}[
	enlargelimits=false,
	width=5.0in,
	height=1.1in,
	%xmin=0,xmax=300,ymin=32,ymax=34,
	xlabel={$k$},
	ylabel={$U2$},
	]
	\addplot[const plot,cyan] file {wykresy/zadanie1_U2.txt};
	\end{axis}
	\end{tikzpicture}

	\caption{Sygna�y wyj�ciowe i wej�ciowe procesu}
	
\end{figure}




\begin{figure}
Jak wida� warto�ci temperatur w punkcie pracy prezentuj� si� nast�puj�co:\\
Y1 = $\num{42,5}\degree$C\\ 
Y2 =  $\num{39}\degree$C\\
	
\end{figure}
	%LAB4
\chapter{Punkt 2}

Na sterowniku zaimplementowali�my zabezpieczenie obiektu przed uszkodzeniem.\\
Je�li odczyt z czujnika T1 \textgreater $\num{150}\degree$C ( czyli w rejestrze D100 \textgreater 15000) wy��czana jest grza�ka 1, czyli do rejestru
D114 jest wpisywane 0.  \\
Analogicznie dla czujnika T1 oraz grza�ki G2.
	%LAB4
\chapter{Punkt 3}

Nastawy PID:\\
(takie jak na labolatorium 3) 	\\
$K1=\num{1,228038}; T_i1=\num{11,602531}; T_d1=\num{1,166250}; $\\
$K2=\num{0,851927}; T_i2=\num{9,777445}; T_d2=\num{0,842038};$\\\\

W algorytmie zosta�y u�yte wcze�niej wyliczone w Matlabie paramtry RR12, RR10, RR10
RR22, RR21, RR20, tak aby zwolni� sterownik z oblicze� i pro�ci� implementacj�.\\\\

Implementacja na sterowniku wraz z zabezpieczeniem opisanym w zadaniu 2.

\begin{lstlisting}
AWARIAY1 := FALSE;
AWARIAY2 := FALSE;
YZAD1 := 4000;
YZAD2 := 4000;

YY1 := INT_TO_REAL(D100);
YY2 := INT_TO_REAL(D102);

IF (YY1 >15000.0) THEN
	AWARIAY1 := TRUE;
END_IF;

IF (YY2 > 15000.0) THEN
	AWARIAY2 := TRUE;
END_IF;

IF (NOT AWARIAY1 AND NOT AWARIAY2) THEN
	YY1 := YY1-Y1pp;
	YY2 := YY2-Y2pp;
	
	EE1 := YZAD1-YY1;
	EE2 := YZAD2-YY2;
	
	du1 := (RR12*e1_2)+(RR11*e1_1)+(RR10*EE1);
	du2 := (RR22*e2_2)+(RR21*e2_1)+(RR20*EE2);
	
	
	UU1 := UU1+du1+U1pp;
	UU2 := UU2+du2+U2pp;
	
	e1_2 := e1_1;
	e2_2 := e2_1;
	
	e1_1 := EE1;
	e2_1 := EE2;
	
	
	D110 := 500;
	D111 := 500;
	
	IF (UU1>1000.0) THEN
		UU1 := 1000.0;
	END_IF;
	
	IF (UU1<0.0) THEN
		UU1 := 0.0;
	END_IF;

END_IF;


IF (UU2>1000.0) THEN
	UU2 := 1000.0;
END_IF;

IF (UU2<0.0) THEN
	UU2 := 0.0;
END_IF;


//ZACHOWANIE W PRZYPADKU AWARII
IF(AWARIAY1) THEN
	UU1 := 0.0;
END_IF;

IF(AWARIAY2) THEN
	UU2 := 0.0;
END_IF;


D114 := REAL_TO_INT(UU1);
D115 := REAL_TO_INT(UU2);
\end{lstlisting}

\begin{figure}
	Na poni�szych wykresach mo�emy zaobserwowa� dzia�anie regulatora dla warto�ci zadanej
	dla obu temperatur, T1 oraz T3 = $\num{50}\degree$C. Pocz�tkowy stan wynosi� ok. T1 = T3 = $\num{25}\degree$C.
\end{figure}


\begin{figure}[ht]
	\centering
	\begin{tikzpicture}
	\begin{axis}[
	enlargelimits=false,
	width=5.0in,
	height=2.5in,
	%xmin=0,xmax=300,
	ymin=2000,ymax=6000,
	%xlabel={$k$},
	ylabel={$Y1$},
	legend pos=south east,
	]
	\addplot[const plot,cyan] file {wykresy/zadanie3_Y1.txt};
	\addplot[red] file {wykresy/zadanie3_Y1zad.txt};
	\legend{$Y1$, $Y1_{zad}$}
	\end{axis}
	\end{tikzpicture}
	
	\begin{tikzpicture}
	\begin{axis}[
	enlargelimits=false,
	width=5.0in,
	height=2.5in,
	%xmin=0,xmax=300,
	ymin=2000,ymax=6000,
	%xlabel={$k$},
	ylabel={$Y2$},
	legend pos=south east,
	]
	\addplot[const plot,cyan] file {wykresy/zadanie3_Y2.txt};
	\addplot[red] file {wykresy/zadanie3_Y2zad.txt};
	\legend{$Y2$, $Y2_{zad}$}
	\end{axis}
	\end{tikzpicture}
	
	\begin{tikzpicture}
	\begin{axis}[
	enlargelimits=false,
	width=5.0in,
	height=1.9in,
	%xmin=0,xmax=300,
	ymin=-100,ymax=1100,
	%xlabel={$k$},
	ylabel={$U1$},
	]
	\addplot[const plot,cyan] file {wykresy/zadanie3_U1.txt};
	\end{axis}
	\end{tikzpicture}
	
	\begin{tikzpicture}
	\begin{axis}[
	enlargelimits=false,
	width=5.0in,
	height=1.9in,
	%xmin=0,xmax=300,
	ymin=-100,ymax=1100,
	xlabel={$k$},
	ylabel={$U2$},
	]
	\addplot[const plot,cyan] file {wykresy/zadanie3_U2.txt};
	\end{axis}
	\end{tikzpicture}
	
	\caption{Sygna�y wyj�ciowe i wej�ciowe procesu}
	
\end{figure}

%	%LAB4
\chapter{Punkt 4}



Zaimplementowali�my algorytmy do regulacji PID i DMC.
W tym punkcie warto doda�, �e przed przyst�pieniem do eksperyment�w na rzeczywistym obiekcie stworzyli�my jego model na podstawie jego odpowiedzi skokowych. 
Dzi�ki temu mogli�my go symulowa� w �rodowisku Matlab i tutaj te� ustali� optymalne parametry algorytm�w regulacji.\\

Do cel�w symulacyjnych przygotowali�my dwa modele naszego obiektu - pierwszy wykorzystuj�c otrzyman� w zadaniu 3 aproksymacj� odpowiedzi skokowej dla pierwszej z liniowych cz�sci charakterystyki statycznej (czyli dla  $\num{0} \leq U \leq \num{50}$) oraz drugi na drugiej liniowej cze�ci tej�e charakterystyki ( $\num{50} \leq U \leq \num{100}$).\\\\

Nast�pnie wyznaczyli�my korzystaj�c z optymalizatora
 \verb|ga|
(ograniczenia lb=[$\num{0,01}$;$\num{1}$;$\num{0,01}$]; 
ub = [$\num{50}$;$\num{50}$;$\num{50}$]) a nast�pnie eksperymentalnie dostrajaj�c parametry algorytm�w PID i DMC ka�dego z liniowych przedzia��w charakterystyki statycznej. Otrzymane rezultaty:\\\\

dla przedzia�u  $\num{0} \leq U \leq \num{50}$:\\
PID:\\
K=$\num{0,14661}$; $T_i$=$\num{1,63138}$; $T_d$=$\num{0,010002}$;\\
DMC:\\
D=$\num{300}$; N=$\num{30}$; $N_u$=$\num{10}$; $\lambda=\num{2,4}$\\\\


dla przedzia�u $\num{50} \leq U \leq \num{100}$:\\
PID:\\
K=$\num{0,194661}$; $T_i$=$\num{1,53138}$; $T_d$=$\num{0,010002}$;\\
DMC:\\
D=$\num{300}$; N=$\num{40}$; $N_u$=$\num{32}$; $\lambda$=$\num{6,86402}$\\\\

Jako�� regulacji oceniali�my na podstawie symulacji przebieg�w regulacji z wykorzystaniem konkretnych modeli dla kilku skok�w warto�ci zadanej. Oto otrzymane rezultaty:\\\\


\begin{figure}[h!]
	\centering
	
	\begin{tikzpicture}
	\begin{axis}[
	width=4.667in,
	height=1.8in,
	enlargelimits=false,
	%xmin=0,xmax=1600,
	ymin=30,ymax=42,
	xlabel={$k$},
	ylabel={$Y$},
	legend pos=north east,
	y tick label style={/pgf/number format/1000 sep=},
	]
	\addplot[const plot,blue] file {wykresy/zadanie4_PID1_Y.txt};
	\addplot[red] file {wykresy/zadanie4_PID1_Yzad.txt};
	\legend{$Y$, $Y_{zad}$}
	\end{axis}
	\end{tikzpicture}
	
	\begin{tikzpicture}
	\begin{axis}[
	width=4.667in,
	height=1.8in,
	enlargelimits=false,
	%xmin=0,xmax=1600,
	ymin=34,ymax=46,
	xlabel={$k$},
	ylabel={$U$},
	legend pos=north east,
	y tick label style={/pgf/number format/1000 sep=},
	]
	\addplot[const plot,blue] file {wykresy/zadanie4_PID1_U.txt};
	\end{axis}
	\end{tikzpicture}
	
	\caption{Przebiegi symulacji dla regulatora PID dla pierwszej liniowej cz�ci charakterystyki statycznej}
\end{figure}


\begin{figure}[h!]
	\centering
	
	\begin{tikzpicture}
	\begin{axis}[
	width=4.667in,
	height=1.8in,
	enlargelimits=false,
	%xmin=0,xmax=1600,
	ymin=30,ymax=42,
	xlabel={$k$},
	ylabel={$Y$},
	legend pos=north east,
	y tick label style={/pgf/number format/1000 sep=},
	]
	\addplot[const plot,blue] file {wykresy/zadanie4_PID2_Y.txt};
	\addplot[red] file {wykresy/zadanie4_PID2_Yzad.txt};
	\legend{$Y$, $Y_{zad}$}
	\end{axis}
	\end{tikzpicture}
	
	\begin{tikzpicture}
	\begin{axis}[
	width=4.667in,
	height=1.8in,
	enlargelimits=false,
	%xmin=0,xmax=1600,
	ymin=34,ymax=46,
	xlabel={$k$},
	ylabel={$U$},
	legend pos=north east,
	y tick label style={/pgf/number format/1000 sep=},
	]
	\addplot[const plot,blue] file {wykresy/zadanie4_PID2_U.txt};
	\end{axis}
	\end{tikzpicture}
	
	\caption{Przebiegi symulacji dla regulatora PID dla drugiej liniowej cz�ci charakterystyki statycznej}
\end{figure}

%%%%%%%%%%%%%%%%%%%%%%%%%%%%%%%%%%%%%%%%%%%%%%%%%%%%%%%%%%%%%%%%%%%%%%%%%%%%%DMC

\begin{figure}[h!]
	\centering
	
	\begin{tikzpicture}
	\begin{axis}[
	width=4.667in,
	height=1.8in,
	enlargelimits=false,
	%xmin=0,xmax=1600,
	ymin=30,ymax=42,
	xlabel={$k$},
	ylabel={$Y$},
	legend pos=north east,
	y tick label style={/pgf/number format/1000 sep=},
	]
	\addplot[const plot,blue] file {wykresy/zadanie4_DMC1_Y.txt};
	\addplot[red] file {wykresy/zadanie4_DMC1_Yzad.txt};
	\legend{$Y$, $Y_{zad}$}
	\end{axis}
	\end{tikzpicture}
	
	\begin{tikzpicture}
	\begin{axis}[
	width=4.667in,
	height=1.8in,
	enlargelimits=false,
	xmin=0,xmax=1000,
	ymin=30,ymax=42,
	xlabel={$k$},
	ylabel={$U$},
	legend pos=north east,
	y tick label style={/pgf/number format/1000 sep=},
	]
	\addplot[const plot,blue] file {wykresy/zadanie4_DMC1_U.txt};
	\end{axis}
	\end{tikzpicture}
	
	\caption{Przebiegi symulacji dla regulatora DMC dla pierwszej liniowej cz�ci charakterystyki statycznej}
\end{figure}


\begin{figure}[h!]
	\centering
	
	\begin{tikzpicture}
	\begin{axis}[
	width=4.667in,
	height=1.8in,
	enlargelimits=false,
	%xmin=0,xmax=1600,
	ymin=30,ymax=42,
	xlabel={$k$},
	ylabel={$Y$},
	legend pos=north east,
	y tick label style={/pgf/number format/1000 sep=},
	]
	\addplot[const plot,blue] file {wykresy/zadanie4_DMC2_Y.txt};
	\addplot[red] file {wykresy/zadanie4_DMC2_Yzad.txt};
	\legend{$Y$, $Y_{zad}$}
	\end{axis}
	\end{tikzpicture}
	
	\begin{tikzpicture}
	\begin{axis}[
	width=4.667in,
	height=1.8in,
	enlargelimits=false,
	xmin=0,xmax=1000,
	ymin=30,ymax=42,
	xlabel={$k$},
	ylabel={$U$},
	legend pos=north east,
	y tick label style={/pgf/number format/1000 sep=},
	]
	\addplot[const plot,blue] file {wykresy/zadanie4_DMC2_U.txt};
	\legend{$U$}
	\end{axis}
	\end{tikzpicture}
	
	\caption{Przebiegi symulacji dla regulatora DMC dla drugiej liniowej cz�ci charakterystyki statycznej}
\end{figure}


	%LAB4
\chapter{Punkt 5}

W tym zadaniu wykonali�my eksperyment z uzyciem pierwszego regulatora PID obliczonego w zadaniu poprzednim. Niestety, w zwi�zku z nieliniowo�ci� obiektu, regulacja nie daje zadowalaj�cych rezultat�w. Regulator PID to regulator liniowy, kt�ry mo�na zastosowa� jedynie do 
obiekt�w o charakterystyce liniowej.\\

W zwi�zku z ograniczeniem czasowym zaj�� labolatoryjnych oraz d�ugim czasem stabilizacji obiektu po skoku sterowania (oko�o 300 sekund), zdecydowali�my si� zastosowa� mniej skok�w sterowania ni� zalecane by�o w poleceniu.  

B��d skumulowany w tym przypadku wyni�s�: $\num{3762,4}$\\\\


\begin{figure}[h!]
	\centering
	
	\begin{tikzpicture}
	\begin{axis}[
	width=4.667in,
	height=1.8in,
	enlargelimits=false,
	%xmin=0,xmax=1600,
	ymin=25,ymax=55,
	xlabel={$k$},
	ylabel={$Y$},
	legend pos=north east,
	y tick label style={/pgf/number format/1000 sep=},
	]
	\addplot[const plot,blue] file {wykresy/zadanie5_PID_Y.txt};
	\addplot[red] file {wykresy/zadanie5_PID_Yzad.txt};
	\legend{$Y$, $Y_{zad}$}
	\end{axis}
	\end{tikzpicture}
	
	\begin{tikzpicture}
	\begin{axis}[
	width=4.667in,
	height=1.8in,
	enlargelimits=false,
	%xmin=0,xmax=1600,
	ymin=-5,ymax=105,
	xlabel={$k$},
	ylabel={$U$},
	legend pos=north east,
	y tick label style={/pgf/number format/1000 sep=},
	]
	\addplot[const plot,blue] file {wykresy/zadanie5_PID_U.txt};
	\end{axis}
	\end{tikzpicture}
	
		\begin{tikzpicture}
	\begin{axis}[
	width=4.667in,
	height=1.8in,
	enlargelimits=false,
	xmin=0,xmax=1050,
	ymin=-5,ymax=20,
	xlabel={$k$},
	ylabel={$E$},
	legend pos=north east,
	y tick label style={/pgf/number format/1000 sep=},
	]
	\addplot[const plot,blue] file {wykresy/zadanie5_PID_blad.txt};
	\end{axis}
	\end{tikzpicture}
	
	\caption{Przebiegi sygna��w uzyskanych podczas eksperymentu na rzeczywistym obiekcie w przypadku zastosowania 1 regulatora PID}
\end{figure}

%	%LAB4
\chapter{Punkt 6}

W tym zadaniu wykorzystali�my technik� regulacji rozmytej. Wykonali�my eksperymenty dla 2 lokalnych regulator�w PID i 2 lokalnych regulator�w DMC. Parametry lokalnych regulator�w dobrali�my takie jak s� przedstawione w zadaniu 4, czyli na podstawie modelowania oraz symulacji liniowych kawa�k�w charakterystyki statycznej.\\


Na poni�ej zamieszczonych przebiegach wida� (dla PID), �e regulacja przebiega znacznie lepiej, ni� przy zastosowaniu klasycznemgo (liniowego) podej�cia. Regulator oscyluje du�o mniej i zbiega do warto�ci zadanej. Lepsze wyniki regulacji mo�na r�wnie� stwierdzi� bo wykresie przedstawiaj�cym b��d. Niestety podczas labolatorium nie uda�o nam si� zastosowa� 3 lub 4 regulator�w lokalnych,
lecz po do�wiadczeniach na projekcie mo�emy stwierdzi�, �e regulacja zapewne by�aby du�o lepsza, co wynika z tego, �e w punkcie przegi�cia istnieje r�wnie� nieliniowo��, z kt�r� nie radz� sobie tylko 2 regulatory w tym oraz nast�pnym zadaniu.


\begin{figure}[h!]
	\centering
	
	\begin{tikzpicture}
	\begin{axis}[
	width=4.667in,
	height=1.8in,
	enlargelimits=false,
	%xmin=0,xmax=1600,
	ymin=25,ymax=55,
	xlabel={$k$},
	ylabel={$Y$},
	legend pos=north east,
	y tick label style={/pgf/number format/1000 sep=},
	]
	\addplot[const plot,blue] file {wykresy/zadanie6_2reg_PID_Y.txt};
	\addplot[red] file {wykresy/zadanie6_2reg_PID_Yzad.txt};
	\legend{$Y$, $Y_{zad}$}
	\end{axis}
	\end{tikzpicture}
	
	\begin{tikzpicture}
	\begin{axis}[
	width=4.667in,
	height=1.8in,
	enlargelimits=false,
	%xmin=0,xmax=1600,
	ymin=-5,ymax=105,
	xlabel={$k$},
	ylabel={$U$},
	legend pos=north east,
	y tick label style={/pgf/number format/1000 sep=},
	]
	\addplot[const plot,blue] file {wykresy/zadanie6_2reg_PID_U.txt};
	\end{axis}
	\end{tikzpicture}
	
	\begin{tikzpicture}
	\begin{axis}[
	width=4.667in,
	height=1.8in,
	enlargelimits=false,
	%xmin=0,xmax=1600,
	ymin=-5,ymax=20,
	xlabel={$k$},
	ylabel={$E$},
	legend pos=north east,
	y tick label style={/pgf/number format/1000 sep=},
	]
	\addplot[const plot,blue] file {wykresy/zadanie6_PID_2reg_blad.txt};
	\end{axis}
	\end{tikzpicture}
	
	\caption{Przebiegi sygna��w uzyskanych podczas eksperymentu na rzeczywistym obiekcie w przypadku zastosowania dw�ch regulator�w PID}
\end{figure}


	%LAB4
\chapter{Punkt 7}

Uda�o nam si� skonfigurowa� po��czenie obiektu Inteco - serwomechanizmu ze sterwonikiem.\\\\

Konfiguracj� oparli�my o tabelk� wej�� i wyj�c zamieszczon� na stronie przedmiotu jak i przedstawion� poni�ej:\\\\
AIN1 - potencjometr analogowy - sygna� pomiarowy z potencjometru (zadajnik po�o�enia, pr�dko�ci k�towej itp.).\\
AIN2 - Tacho Analogowy - sygna� pomiarowy pr�dko�ci obrotowej z tachopr�dnicy.\\
X2  - (Therm) Flaga limitu temperatury.\\
X1 - (EncB1) Enkoder inkrementalny, fala B, o� pozioma.\\
X0 - (EncA1) Enkoder inkrementalny, fala A, o� pozioma.\\\\

PWM  - Sygna� steruj�cy typu PWM dla silnika DC.\\
Y0  - Zalecana cz�stotliwo�� tego sygna�u mie�ci si� w przedziale (5-15) kHz.\\
Y1  - (Brake) Sygna� zatrzymuj�cy prac� silnika DC.\\
Y2  - (Dir0) Sygna� zmiany kierunku obrot�w silnika DC.\\\\

Wymagane by�o r�wnie� dodanie wej�cia analogowego w zak�adce Module Configuration. Post�puj�c zgodnie
z punktem 3.9 pliku "Laboratorium 5 � instrukcja do obiekt�w Inteco" uda�o nam si� pod��czy� obiekt.\\
Nast�pnie skonfigurowali�my wej�cie licznikowe - pomiar czestotliwo�ci oraz PWM dla wej�� Y0 - sterowanie tachopr�dnicy, Y1 - sygna� hamuj�cy, r�wnie� w spos�b opisany w instrukcji.\\\\

Wy��czaj�c hamowanie ( podanie 0 na pole EN bloczka PWM dla Y1) oraz manipuluj�c warto�ci� parametru K100 bloczka 
PWM dla Y0 uda�o nam si� wprawi� w ruch elementy obrotowe serwa.\\

Dowodem na ten fakt jest za��czony do sprawozdanie film.


%	%LAB4
\chapter{Punkt 8}





	

\end{document}


