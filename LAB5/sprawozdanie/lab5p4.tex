%LAB4
\chapter{Punkt 4}



Zaimplementowali�my algorytmy do regulacji PID i DMC.
W tym punkcie warto doda�, �e przed przyst�pieniem do eksperyment�w na rzeczywistym obiekcie stworzyli�my jego model na podstawie jego odpowiedzi skokowych. 
Dzi�ki temu mogli�my go symulowa� w �rodowisku Matlab i tutaj te� ustali� optymalne parametry algorytm�w regulacji.\\

Do cel�w symulacyjnych przygotowali�my dwa modele naszego obiektu - pierwszy wykorzystuj�c otrzyman� w zadaniu 3 aproksymacj� odpowiedzi skokowej dla pierwszej z liniowych cz�sci charakterystyki statycznej (czyli dla  $\num{0} \leq U \leq \num{50}$) oraz drugi na drugiej liniowej cze�ci tej�e charakterystyki ( $\num{50} \leq U \leq \num{100}$).\\\\

Nast�pnie wyznaczyli�my korzystaj�c z optymalizatora
 \verb|ga|
(ograniczenia lb=[$\num{0,01}$;$\num{1}$;$\num{0,01}$]; 
ub = [$\num{50}$;$\num{50}$;$\num{50}$]) a nast�pnie eksperymentalnie dostrajaj�c parametry algorytm�w PID i DMC ka�dego z liniowych przedzia��w charakterystyki statycznej. Otrzymane rezultaty:\\\\

dla przedzia�u  $\num{0} \leq U \leq \num{50}$:\\
PID:\\
K=$\num{0,14661}$; $T_i$=$\num{1,63138}$; $T_d$=$\num{0,010002}$;\\
DMC:\\
D=$\num{300}$; N=$\num{30}$; $N_u$=$\num{10}$; $\lambda=\num{2,4}$\\\\


dla przedzia�u $\num{50} \leq U \leq \num{100}$:\\
PID:\\
K=$\num{0,194661}$; $T_i$=$\num{1,53138}$; $T_d$=$\num{0,010002}$;\\
DMC:\\
D=$\num{300}$; N=$\num{40}$; $N_u$=$\num{32}$; $\lambda$=$\num{6,86402}$\\\\

Jako�� regulacji oceniali�my na podstawie symulacji przebieg�w regulacji z wykorzystaniem konkretnych modeli dla kilku skok�w warto�ci zadanej. Oto otrzymane rezultaty:\\\\


\begin{figure}[h!]
	\centering
	
	\begin{tikzpicture}
	\begin{axis}[
	width=4.667in,
	height=1.8in,
	enlargelimits=false,
	%xmin=0,xmax=1600,
	ymin=30,ymax=42,
	xlabel={$k$},
	ylabel={$Y$},
	legend pos=north east,
	y tick label style={/pgf/number format/1000 sep=},
	]
	\addplot[const plot,blue] file {wykresy/zadanie4_PID1_Y.txt};
	\addplot[red] file {wykresy/zadanie4_PID1_Yzad.txt};
	\legend{$Y$, $Y_{zad}$}
	\end{axis}
	\end{tikzpicture}
	
	\begin{tikzpicture}
	\begin{axis}[
	width=4.667in,
	height=1.8in,
	enlargelimits=false,
	%xmin=0,xmax=1600,
	ymin=34,ymax=46,
	xlabel={$k$},
	ylabel={$U$},
	legend pos=north east,
	y tick label style={/pgf/number format/1000 sep=},
	]
	\addplot[const plot,blue] file {wykresy/zadanie4_PID1_U.txt};
	\end{axis}
	\end{tikzpicture}
	
	\caption{Przebiegi symulacji dla regulatora PID dla pierwszej liniowej cz�ci charakterystyki statycznej}
\end{figure}


\begin{figure}[h!]
	\centering
	
	\begin{tikzpicture}
	\begin{axis}[
	width=4.667in,
	height=1.8in,
	enlargelimits=false,
	%xmin=0,xmax=1600,
	ymin=30,ymax=42,
	xlabel={$k$},
	ylabel={$Y$},
	legend pos=north east,
	y tick label style={/pgf/number format/1000 sep=},
	]
	\addplot[const plot,blue] file {wykresy/zadanie4_PID2_Y.txt};
	\addplot[red] file {wykresy/zadanie4_PID2_Yzad.txt};
	\legend{$Y$, $Y_{zad}$}
	\end{axis}
	\end{tikzpicture}
	
	\begin{tikzpicture}
	\begin{axis}[
	width=4.667in,
	height=1.8in,
	enlargelimits=false,
	%xmin=0,xmax=1600,
	ymin=34,ymax=46,
	xlabel={$k$},
	ylabel={$U$},
	legend pos=north east,
	y tick label style={/pgf/number format/1000 sep=},
	]
	\addplot[const plot,blue] file {wykresy/zadanie4_PID2_U.txt};
	\end{axis}
	\end{tikzpicture}
	
	\caption{Przebiegi symulacji dla regulatora PID dla drugiej liniowej cz�ci charakterystyki statycznej}
\end{figure}

%%%%%%%%%%%%%%%%%%%%%%%%%%%%%%%%%%%%%%%%%%%%%%%%%%%%%%%%%%%%%%%%%%%%%%%%%%%%%DMC

\begin{figure}[h!]
	\centering
	
	\begin{tikzpicture}
	\begin{axis}[
	width=4.667in,
	height=1.8in,
	enlargelimits=false,
	%xmin=0,xmax=1600,
	ymin=30,ymax=42,
	xlabel={$k$},
	ylabel={$Y$},
	legend pos=north east,
	y tick label style={/pgf/number format/1000 sep=},
	]
	\addplot[const plot,blue] file {wykresy/zadanie4_DMC1_Y.txt};
	\addplot[red] file {wykresy/zadanie4_DMC1_Yzad.txt};
	\legend{$Y$, $Y_{zad}$}
	\end{axis}
	\end{tikzpicture}
	
	\begin{tikzpicture}
	\begin{axis}[
	width=4.667in,
	height=1.8in,
	enlargelimits=false,
	xmin=0,xmax=1000,
	ymin=30,ymax=42,
	xlabel={$k$},
	ylabel={$U$},
	legend pos=north east,
	y tick label style={/pgf/number format/1000 sep=},
	]
	\addplot[const plot,blue] file {wykresy/zadanie4_DMC1_U.txt};
	\end{axis}
	\end{tikzpicture}
	
	\caption{Przebiegi symulacji dla regulatora DMC dla pierwszej liniowej cz�ci charakterystyki statycznej}
\end{figure}


\begin{figure}[h!]
	\centering
	
	\begin{tikzpicture}
	\begin{axis}[
	width=4.667in,
	height=1.8in,
	enlargelimits=false,
	%xmin=0,xmax=1600,
	ymin=30,ymax=42,
	xlabel={$k$},
	ylabel={$Y$},
	legend pos=north east,
	y tick label style={/pgf/number format/1000 sep=},
	]
	\addplot[const plot,blue] file {wykresy/zadanie4_DMC2_Y.txt};
	\addplot[red] file {wykresy/zadanie4_DMC2_Yzad.txt};
	\legend{$Y$, $Y_{zad}$}
	\end{axis}
	\end{tikzpicture}
	
	\begin{tikzpicture}
	\begin{axis}[
	width=4.667in,
	height=1.8in,
	enlargelimits=false,
	xmin=0,xmax=1000,
	ymin=30,ymax=42,
	xlabel={$k$},
	ylabel={$U$},
	legend pos=north east,
	y tick label style={/pgf/number format/1000 sep=},
	]
	\addplot[const plot,blue] file {wykresy/zadanie4_DMC2_U.txt};
	\legend{$U$}
	\end{axis}
	\end{tikzpicture}
	
	\caption{Przebiegi symulacji dla regulatora DMC dla drugiej liniowej cz�ci charakterystyki statycznej}
\end{figure}

