%LAB4
\chapter{Punkt 5}

W tym zadaniu wykonali�my eksperyment z uzyciem pierwszego regulatora PID obliczonego w zadaniu poprzednim. Niestety, w zwi�zku z nieliniowo�ci� obiektu, regulacja nie daje zadowalaj�cych rezultat�w. Regulator PID to regulator liniowy, kt�ry mo�na zastosowa� jedynie do 
obiekt�w o charakterystyce liniowej.\\

W zwi�zku z ograniczeniem czasowym zaj�� labolatoryjnych oraz d�ugim czasem stabilizacji obiektu po skoku sterowania (oko�o 300 sekund), zdecydowali�my si� zastosowa� mniej skok�w sterowania ni� zalecane by�o w poleceniu.  

B��d skumulowany w tym przypadku wyni�s�: $\num{3762,4}$\\\\


\begin{figure}[h!]
	\centering
	
	\begin{tikzpicture}
	\begin{axis}[
	width=4.667in,
	height=1.8in,
	enlargelimits=false,
	%xmin=0,xmax=1600,
	ymin=25,ymax=55,
	xlabel={$k$},
	ylabel={$Y$},
	legend pos=north east,
	y tick label style={/pgf/number format/1000 sep=},
	]
	\addplot[const plot,blue] file {wykresy/zadanie5_PID_Y.txt};
	\addplot[red] file {wykresy/zadanie5_PID_Yzad.txt};
	\legend{$Y$, $Y_{zad}$}
	\end{axis}
	\end{tikzpicture}
	
	\begin{tikzpicture}
	\begin{axis}[
	width=4.667in,
	height=1.8in,
	enlargelimits=false,
	%xmin=0,xmax=1600,
	ymin=-5,ymax=105,
	xlabel={$k$},
	ylabel={$U$},
	legend pos=north east,
	y tick label style={/pgf/number format/1000 sep=},
	]
	\addplot[const plot,blue] file {wykresy/zadanie5_PID_U.txt};
	\end{axis}
	\end{tikzpicture}
	
		\begin{tikzpicture}
	\begin{axis}[
	width=4.667in,
	height=1.8in,
	enlargelimits=false,
	xmin=0,xmax=1050,
	ymin=-5,ymax=20,
	xlabel={$k$},
	ylabel={$E$},
	legend pos=north east,
	y tick label style={/pgf/number format/1000 sep=},
	]
	\addplot[const plot,blue] file {wykresy/zadanie5_PID_blad.txt};
	\end{axis}
	\end{tikzpicture}
	
	\caption{Przebiegi sygna��w uzyskanych podczas eksperymentu na rzeczywistym obiekcie w przypadku zastosowania 1 regulatora PID}
\end{figure}
