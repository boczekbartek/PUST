%LAB4
\chapter{Punkt 6}

W tym zadaniu wykorzystali�my technik� regulacji rozmytej. Wykonali�my eksperymenty dla 2 lokalnych regulator�w PID i 2 lokalnych regulator�w DMC. Parametry lokalnych regulator�w dobrali�my takie jak s� przedstawione w zadaniu 4, czyli na podstawie modelowania oraz symulacji liniowych kawa�k�w charakterystyki statycznej.\\


Na poni�ej zamieszczonych przebiegach wida� (dla PID), �e regulacja przebiega znacznie lepiej, ni� przy zastosowaniu klasycznemgo (liniowego) podej�cia. Regulator oscyluje du�o mniej i zbiega do warto�ci zadanej. Lepsze wyniki regulacji mo�na r�wnie� stwierdzi� bo wykresie przedstawiaj�cym b��d. Niestety podczas labolatorium nie uda�o nam si� zastosowa� 3 lub 4 regulator�w lokalnych,
lecz po do�wiadczeniach na projekcie mo�emy stwierdzi�, �e regulacja zapewne by�aby du�o lepsza, co wynika z tego, �e w punkcie przegi�cia istnieje r�wnie� nieliniowo��, z kt�r� nie radz� sobie tylko 2 regulatory w tym oraz nast�pnym zadaniu.


\begin{figure}[h!]
	\centering
	
	\begin{tikzpicture}
	\begin{axis}[
	width=4.667in,
	height=1.8in,
	enlargelimits=false,
	%xmin=0,xmax=1600,
	ymin=25,ymax=55,
	xlabel={$k$},
	ylabel={$Y$},
	legend pos=north east,
	y tick label style={/pgf/number format/1000 sep=},
	]
	\addplot[const plot,blue] file {wykresy/zadanie6_2reg_PID_Y.txt};
	\addplot[red] file {wykresy/zadanie6_2reg_PID_Yzad.txt};
	\legend{$Y$, $Y_{zad}$}
	\end{axis}
	\end{tikzpicture}
	
	\begin{tikzpicture}
	\begin{axis}[
	width=4.667in,
	height=1.8in,
	enlargelimits=false,
	%xmin=0,xmax=1600,
	ymin=-5,ymax=105,
	xlabel={$k$},
	ylabel={$U$},
	legend pos=north east,
	y tick label style={/pgf/number format/1000 sep=},
	]
	\addplot[const plot,blue] file {wykresy/zadanie6_2reg_PID_U.txt};
	\end{axis}
	\end{tikzpicture}
	
	\begin{tikzpicture}
	\begin{axis}[
	width=4.667in,
	height=1.8in,
	enlargelimits=false,
	%xmin=0,xmax=1600,
	ymin=-5,ymax=20,
	xlabel={$k$},
	ylabel={$E$},
	legend pos=north east,
	y tick label style={/pgf/number format/1000 sep=},
	]
	\addplot[const plot,blue] file {wykresy/zadanie6_PID_2reg_blad.txt};
	\end{axis}
	\end{tikzpicture}
	
	\caption{Przebiegi sygna��w uzyskanych podczas eksperymentu na rzeczywistym obiekcie w przypadku zastosowania dw�ch regulator�w PID}
\end{figure}

