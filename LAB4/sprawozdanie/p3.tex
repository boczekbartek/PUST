%LAB4
\chapter{Punkt 3}

Paramatry cz�onu inercyjnego drugiego rz�du z op�nieniem zosta�y dobrane w wyniku minimalizacji �redniokwadratowego b��du z u�yciem optymalizatora $ga$.\\\\ Dla odpowiedzi $s$ parametry te wynosz�: $T_1 =  \num{0,0555}$; $T_2 = \num{118,9431}$; $K = \num{0,3926}$ $T_d = \num{10}$.\\\\
Natomiast dla odpowiedzi $s^z$: $T_1 = \num{54,5808}$; $T_2 = \num{60,8558}$; $K=\num{0,14}$;  $T_d = \num{10}$.

\begin{figure}[ht]
	\centering
	\begin{tikzpicture}
	\begin{axis}[
	width=5.0in,
	height=2.4in,
	xmin=0,xmax=320,ymin=-0.01,ymax=0.4,
	xlabel={$k$},
	ylabel={$s$},
	legend pos=south east,
	y tick label style={/pgf/number format/1000 sep=},
	]
	\addplot[const plot,cyan] file {wykresy/zadanie3_s.txt};
	\end{axis}
	\end{tikzpicture}
	\begin{tikzpicture}
	\begin{axis}[
	width=5.0in,
	height=2.4in,
	xmin=0,xmax=320,ymin=-0.1,ymax=0.2,
	xlabel={$k$},
	ylabel={$s^{\mathrm{z}}$},
	legend pos=south east,
	y tick label style={/pgf/number format/1000 sep=},
	]
	\addplot[const plot,cyan] file {wykresy/zadanie3_sz.txt};
	\end{axis}
	\end{tikzpicture}
	\caption{Odpowied� skokowa przy skoku sygna�u steruj�cego (g�ra) oraz zak��caj�cego (d�)}
	\label{odpskokUdmc3}
\end{figure}

\begin{figure}[ht]
	\centering
	\begin{tikzpicture}
	\begin{axis}[
	width=5.0in,
	height=2.4in,
	xmin=0,xmax=320,ymin=-0.01,ymax=0.4,
	xlabel={$k$},
	ylabel={$Y$},
	legend pos=south east,
	y tick label style={/pgf/number format/1000 sep=},
	]
	\addplot[const plot,red] file {wykresy/zadanie3_s.txt};
	\addplot[cyan] file {wykresy/zadanie3_saproks.txt};
	\legend{orygina�,aproksymacja}
	\end{axis}
	\end{tikzpicture}
	\caption{Por�wnanie odpowiedzi skokowej oryginalnej i aproksymowanej dla skoku sygna�u steruj�cego}
	\label{odpskokUdmc4}
\end{figure}

\begin{figure}[ht]
	\centering
	\begin{tikzpicture}
	\begin{axis}[
	width=5.0in,
	height=2.4in,
	xmin=0,xmax=320,ymin=-0.1,ymax=0.2,
	xlabel={$k$},
	ylabel={$Y$},
	legend pos=south east,
	y tick label style={/pgf/number format/1000 sep=},
	]
	\addplot[const plot,red] file {wykresy/zadanie3_sz.txt};
	\addplot[cyan] file {wykresy/zadanie3_szaproks.txt};
	\legend{orygina�,aproksymacja}
	\end{axis}
	\end{tikzpicture}
	\caption{Por�wnanie odpowiedzi skokowej oryginalnej i aproksymowanej dla skoku sygna�u zak��cenia}
	\label{odpskokUdmc1}
\end{figure}

