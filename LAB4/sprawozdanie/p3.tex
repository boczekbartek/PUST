%LAB4
\chapter{Punkt 3}

Uzasadni� wyb�r parametr�w optymalizacji. - czy panujemy nad optymalizatorem ga, czy jestesmy swiadomi jego ograniczen, ile zrobilismy eksperymentow

\begin{comment}
\begin{figure}[ht]
	\centering
	\begin{tikzpicture}
	\begin{axis}[
	enlargelimits=false,
	width=5.0in,
	height=3.4in,
	%xmin=0,xmax=350,ymin=-0.01,ymax=0.4,
	xlabel={$k$},
	ylabel={$odp. skok.$},
	legend pos=south east,
	y tick label style={/pgf/number format/1000 sep=},
	]
	\addplot[const plot,red] file {wykresy/przykladowe/odpowiedzskokowa_T1_G1.txt};
	\end{axis}
	\end{tikzpicture}
	\caption{Odpowied� skokowa}

\end{figure}



\begin{figure}[ht]
	\centering
	\begin{tikzpicture}
	\begin{axis}[
	enlargelimits=false,
	width=5.0in,
	height=3.4in,
	%xmin=0,xmax=350,ymin=-0.01,ymax=0.4,
	xlabel={$k$},
	ylabel={$Y$},
	legend pos=south east,
	y tick label style={/pgf/number format/1000 sep=},
	]
	\addplot[const plot,red] file {wykresy/przykladowe/odpowiedzskokowa_T1_G1.txt};
	\addplot[cyan] file {wykresy/przykladowe/funkcja_aproksymujaca_T1_G1.txt};
	\legend{orygina�,aproksymacja}
	\end{axis}
	\end{tikzpicture}
	\caption{Por�wnanie odpowiedzi skokowej wykorzystywanej w alogorytmie DMC i jej aproksymacji}
\end{figure}

\end{comment}
