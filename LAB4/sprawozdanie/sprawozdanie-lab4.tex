%LAB4
\documentclass[a4paper,titlepage,11pt,twosides,floatssmall]{mwrep}
\usepackage[left=2.5cm,right=2.5cm,top=2.5cm,bottom=2.5cm]{geometry}
\usepackage[OT1]{fontenc}
\usepackage{polski}
\usepackage{amsmath}
\usepackage{amsfonts}
\usepackage{amssymb}
\usepackage{graphicx}
\usepackage{url}
\usepackage{tikz}
\usetikzlibrary{arrows,calc,decorations.markings,math,arrows.meta}
\usepackage{rotating}
\usepackage[percent]{overpic}
\usepackage[cp1250]{inputenc}
\usepackage{xcolor}
\usepackage{pgfplots}
\usetikzlibrary{pgfplots.groupplots}
\usepackage{listings}
\usepackage{matlab-prettifier}
\usepackage{siunitx}
\usepackage{placeins}
\usepackage{gensymb}
\usepackage{verbatim}
\definecolor{szary}{rgb}{0.95,0.95,0.95}
\sisetup{detect-weight,exponent-product=\cdot,output-decimal-marker={,},per-mode=symbol,binary-units=true,range-phrase={-},range-units=single}
\SendSettingsToPgf

%konfiguracje pakietu listings
\lstset{
	backgroundcolor=\color{szary},
	frame=single,
	breaklines=true,
}
\lstdefinestyle{customlatex}{
	basicstyle=\footnotesize\ttfamily,
	%basicstyle=\small\ttfamily,
}
\lstdefinestyle{customc}{
	breaklines=true,
	frame=tb,
	language=C,
	xleftmargin=0pt,
	showstringspaces=false,
	basicstyle=\small\ttfamily,
	keywordstyle=\bfseries\color{green!40!black},
	commentstyle=\itshape\color{purple!40!black},
	identifierstyle=\color{blue},
	stringstyle=\color{orange},
}
\lstdefinestyle{custommatlab}{
	captionpos=t,
	breaklines=true,
	frame=tb,
	xleftmargin=0pt,
	language=matlab,
	showstringspaces=false,
	%basicstyle=\footnotesize\ttfamily,
	basicstyle=\scriptsize\ttfamily,
	keywordstyle=\bfseries\color{green!40!black},
	commentstyle=\itshape\color{purple!40!black},
	identifierstyle=\color{blue},
	stringstyle=\color{orange},
}

%wymiar tekstu (bez �ywej paginy)
\textwidth 160mm \textheight 247mm

%ustawienia pakietu pgfplots
\pgfplotsset{
	tick label style={font=\scriptsize},
	label style={font=\small},
	legend style={font=\small},
	title style={font=\small}
}

\def\figurename{Rys.}
\def\tablename{Tab.}

%konfiguracja liczby p�ywaj�cych element�w
\setcounter{topnumber}{0}%2
\setcounter{bottomnumber}{3}%1
\setcounter{totalnumber}{5}%3
\renewcommand{\textfraction}{0.01}%0.2
\renewcommand{\topfraction}{0.95}%0.7
\renewcommand{\bottomfraction}{0.95}%0.3
\renewcommand{\floatpagefraction}{0.35}%0.5

\begin{document}
	\frenchspacing
	\pagestyle{uheadings}
	
	%strona tytu�owa
	\title{\bf Sprawozdanie z �wiczenia laboratoryjnego nr 4\vskip 0.1cm}
	\author{Bart�omiej Boczek, Aleksander Piotrowski, �ukasz �migielski}
	\date{7 maja 2017}
	
	\makeatletter
	\renewcommand{\maketitle}{\begin{titlepage}
			\begin{center}{\LARGE {\bf
						Wydzia� Elektroniki i Technik Informacyjnych}}\\
				\vspace{0.4cm}
				{\LARGE {\bf Politechnika Warszawska}}\\
				\vspace{0.3cm}
			\end{center}
			\vspace{5cm}
			\begin{center}
				{\bf \LARGE Projektowanie uk�ad�w sterowania\\ (projekt grupowy) \vskip 0.1cm}
			\end{center}
			\vspace{1cm}
			\begin{center}
				{\bf \LARGE \@title}
			\end{center}
			\vspace{2cm}
			\begin{center}
				{\bf \Large \@author \par}
			\end{center}
			\vspace*{\stretch{6}}
			\begin{center}
				\bf{\large{Warszawa, \@date\vskip 0.1cm}}
			\end{center}
		\end{titlepage}
	}
	\makeatother
	
	\maketitle
	
	\tableofcontents
	%LAB2
\chapter{Punkt 1 }

Dla warto�ci $U_{\mathrm{pp}} = \num{31}$ oraz $Z = \num{0}$ warto�� wyj�cia stabilizuje si� na warto�ci $\num{31,5}\degree C$, co �wiadczy o tym, �e jest to punkt pracy uk�adu.


	%PROJ4
\chapter{Punkt 2}



\begin{figure}[ht]
	\centering
		
	\begin{tikzpicture}
	\begin{axis}[
	enlargelimits=false,
	width=4.667in,
	height=4in,
	%xmin=0,xmax=350
	ymin=-2.8,ymax=0.2,
	xlabel={$k$},
	ylabel={$Y$},
	legend pos=south east,
	y tick label style={/pgf/number format/1000 sep=},
	]
	\addplot[const plot,green] file {../wykresy/zad2odpowiedz_skokowa_Y_-1.txt};
	\addplot[const plot,yellow] file {../wykresy/zad2odpowiedz_skokowa_Y_-0.5.txt};
	\addplot[const plot,gray] file {../wykresy/zad2odpowiedz_skokowa_Y_0.txt};
	\addplot[const plot,red] file {../wykresy/zad2odpowiedz_skokowa_Y_0.5.txt};
	\addplot[const plot,blue] file {../wykresy/zad2odpowiedz_skokowa_Y_1.txt};
	\legend{$U=\num{-1}$,
			$U=\num{-0,5}$,
			$U=\num{0}$,
			$U=\num{0,5}$,
			$U=\num{1}$}

	\end{axis}
	\end{tikzpicture}
	
		\begin{tikzpicture}
	\begin{axis}[
	enlargelimits=false,
	width=4.667in,
	height=1.5in,
	%xmin=0,xmax=350
	%ymin=34,ymax=52,
	xlabel={$k$},
	ylabel={$Y$},
	legend pos=south east,
	y tick label style={/pgf/number format/1000 sep=},
	]
	\addplot[const plot,green] file {../wykresy/zad2odpowiedz_skokowa_U_-1.txt};
	\addplot[const plot,yellow] file {../wykresy/zad2odpowiedz_skokowa_U_-0.5.txt};
	\addplot[const plot,gray] file {../wykresy/zad2odpowiedz_skokowa_U_0.txt};
	\addplot[const plot,red] file {../wykresy/zad2odpowiedz_skokowa_U_0.5.txt};
	\addplot[const plot,blue] file {../wykresy/zad2odpowiedz_skokowa_U_1.txt};		
		\end{axis}
		\end{tikzpicture}
	
	\caption{Odpowiedzi skokowe procesu dla pi�ciu r�nych zmian sygna�u steruj�cego}
	
\end{figure}







\begin{figure}[ht]
	\centering
	
	\begin{tikzpicture}
	\begin{axis}[
	enlargelimits=false,
	width=4.667in,
	height=3in,
	%xmin=0,xmax=350
	%ymin=-2.8,ymax=0.2,
	xlabel={$U$},
	ylabel={$Y$},
	legend pos=south east,
	y tick label style={/pgf/number format/1000 sep=},
	]
	\addplot[const plot,green] file {../wykresy/char_stat.txt};
	\end{axis}
	\end{tikzpicture}
	
	\caption{Charakterystyka statyczna}
	
\end{figure}


\begin{figure}
	
	W�a�ciwo�ci statyczne obiektu nie s� w ca�o�ciowym uj�ciu- wzgl�dem dziedziny U - liniowe.
	Charakterystyka statyczna to z�o�enie dw�ch prostych, pierwszej szybko rosn�cej, drugiej bardzo wolno, niemal stale, z punktem przegi�cia w okolicach $\num{-0,1}$(u)
	W�a�ciwo�ci dynamiczne obiektu: uk�ad jest stabilny, ma op�nienie w ilo�ci 5 chwil k, ma jedn� lub dwie inercje; wszystkie te cechy dynamiczne mo�na uzna� za w przybli�eniu liniowe.
\end{figure}
	%PROJ4
\chapter{Punkt 3 i 4}

Symulacja trwa 1000 chwil. Skoki:\\
- k=21, yzad = -1\\
- k=201, yzad = -2\\
- k=401, yzad = -0,1\\
- k=601, yzad = -2\\
- k=801, yzad = -1\\\\

PID: b��d E = 133,2526, parametry: K=0,17; Ti=3,5; Td=1,1;\\
DMC: b��d E = 236.8488, parametry: D=N=Nu=50; lambda=250;\\
Parametry dobrane eksperymentalnie.\\\\

Widzimy �e dla skok�w warto�ci zadanej zawieraj�cych si� w pierwszej, szybszej cz�ci charakterystyki statycznej
regulatory dzia�aj� du�o lepiej ni� w przypadku skok�w zawieraj�cych si� w drugiej, bardzo wolnej cz�ci charakterystyki. 
200 chwil k to za ma�o aby obiekt osi�gn�� warto�� zadan� podczas gdy osi�gni�cie "szybszej" warto�ci zadanej zajmuje mu kilkadziesi�t chwil.\\
Strojenie liniowego regulatora w tym przypadku przy pomocy optymalizator�w zapewne da�oby lepszy efekt, ale nadal nieliniowo�� obiektu mocno by "dokucza�a" i nie by�aby to tak dobra regulacja jak w przypadku obiektu liniowego.\\


\begin{figure}[h!]
	\centering
	
	\begin{tikzpicture}
	\begin{axis}[
	width=4.667in,
	height=1.8in,
	enlargelimits=false,
	%xmin=0,xmax=1600,
	%ymin=-0.5,ymax=1,
	xlabel={$k$},
	ylabel={$Y$},
	legend pos=north east,
	y tick label style={/pgf/number format/1000 sep=},
	]
	\addplot[const plot,blue] file {../wykresy/zad34pid_Y.txt};
	\addplot[red] file {../wykresy/yzad.txt};
	\legend{$Y$, $Y_{zad}$}
	\end{axis}
	\end{tikzpicture}
	
	\begin{tikzpicture}
	\begin{axis}[
	width=4.667in,
	height=1.8in,
	enlargelimits=false,
	%xmin=0,xmax=1600,
	%ymin=-0.5,ymax=1,
	xlabel={$k$},
	ylabel={$U$},
	legend pos=north east,
	y tick label style={/pgf/number format/1000 sep=},
	]
	\addplot[const plot,blue] file {../wykresy/zad34pid_U.txt};
	\end{axis}
	\end{tikzpicture}
	
	
	\caption{Przebiegi sygna��w dla PID}
\end{figure}


\begin{figure}[h!]
	\centering
	
	\begin{tikzpicture}
	\begin{axis}[
	width=4.667in,
	height=1.8in,
	enlargelimits=false,
	%xmin=0,xmax=1600,
	%ymin=-0.5,ymax=1,
	xlabel={$k$},
	ylabel={$Y$},
	legend pos=north east,
	y tick label style={/pgf/number format/1000 sep=},
	]
	\addplot[const plot,blue] file {../wykresy/zad34dmc_Y.txt};
	\addplot[red] file {../wykresy/yzad.txt};
	\legend{$Y$, $Y_{zad}$}
	\end{axis}
	\end{tikzpicture}
	
	\begin{tikzpicture}
	\begin{axis}[
	width=4.667in,
	height=1.8in,
	enlargelimits=false,
	%xmin=0,xmax=1600,
	%ymin=-5,ymax=105,
	xlabel={$k$},
	ylabel={$U$},
	legend pos=north east,
	y tick label style={/pgf/number format/1000 sep=},
	]
	\addplot[const plot,blue] file {../wykresy/zad34dmc_U.txt};
	\end{axis}
	\end{tikzpicture}
	
	
	\caption{Przebiegi sygna��w dla DMC}
\end{figure}

	\chapter{Punkt 4}

/*sample text\\
Parametry regulatora DMC: $D=\num{110}$; $N=\num{130}$; $N_u=\num{130}$; $\lambda=\num{0,92}$ 

$\num{0}$ $\le$ $G1(k)$ $\le$ $\num{100}$\\
sample text*/


	%PROJ3
\chapter{Punkt 5}

%%%%%%%%%%%%%%%%%%%%%%%%%%%%%%%%%%%%%%%%%%%%%%%%%%%%%%%%%%%%%%%%%%%%%%V1
\begin{figure}[ht]
	\centering
	\begin{tikzpicture}
	\begin{axis}[
	width=7.0in,
	height=2.4in,
	xmin=0,xmax=1600,ymin=-1,ymax=4,
	xlabel={$k$},
	ylabel={$y_1$},
	legend pos=south east,
	y tick label style={/pgf/number format/1000 sep=},
	]
	\addplot[const plot,red, densely dashed] file {wykresy/zadanie5_PID_V1_Y1_zad.txt};
	\addplot[const plot,cyan] file {wykresy/zadanie5_PID_V1_Y1.txt};
	\legend{$y_1^{zad}$, $y_1$}
	\end{axis}
	\end{tikzpicture}
	
	\caption{Sygna� wyj�ciowy $y_1$ dla V1 (CO TO??)}
	\label{odpskokUdmc3}
\end{figure}

\begin{figure}[ht]
	\centering
	\begin{tikzpicture}
	\begin{axis}[
	width=7.0in,
	height=2.4in,
	xmin=0,xmax=1600,ymin=-1,ymax=4,
	xlabel={$k$},
	ylabel={$y_2$},
	legend pos=south east,
	y tick label style={/pgf/number format/1000 sep=},
	]
	\addplot[const plot,red, densely dashed] file {wykresy/zadanie5_PID_V1_Y2_zad.txt};
	\addplot[const plot,cyan] file {wykresy/zadanie5_PID_V1_Y2.txt};
	\legend{$y_2^{zad}$, $y_2$}
	\end{axis}
	\end{tikzpicture}
	
	\caption{Sygna� wyj�ciowy $y_1$ dla V1 (CO TO??)}
	\label{odpskokUdmc3}
\end{figure}

\begin{figure}[ht]
	\centering
	\begin{tikzpicture}
	\begin{axis}[
	width=7.0in,
	height=2.4in,
	xmin=0,xmax=1600,ymin=-5,ymax=5,
	xlabel={$k$},
	ylabel={$u_1$},
	legend pos=south east,
	y tick label style={/pgf/number format/1000 sep=},
	]
	\addplot[const plot,cyan] file {wykresy/zadanie5_PID_V1_U1.txt};
	\end{axis}
	\end{tikzpicture}
	
	\caption{Sygna� wej�ciowy $u_1$ dla V1 (CO TO??)}
	\label{odpskokUdmc3}
\end{figure}




\begin{figure}[ht]
	\centering
	\begin{tikzpicture}
	\begin{axis}[
	width=7.0in,
	height=2.4in,
	xmin=0,xmax=1600,ymin=-2,ymax=2,
	xlabel={$k$},
	ylabel={$u_2$},
	legend pos=south east,
	y tick label style={/pgf/number format/1000 sep=},
	]
	\addplot[const plot,cyan] file {wykresy/zadanie5_PID_V1_U2.txt};
	\end{axis}
	\end{tikzpicture}
	
	\caption{Sygna� wej�ciowy $u_2$ dla V1 (CO TO??)}
	\label{odpskokUdmc3}
\end{figure}

\begin{figure}[ht]
	\centering
	\begin{tikzpicture}
	\begin{axis}[
	width=7.0in,
	height=2.4in,
	xmin=0,xmax=1600,ymin=-2,ymax=2,
	xlabel={$k$},
	ylabel={$e_1$},
	legend pos=south east,
	y tick label style={/pgf/number format/1000 sep=},
	]
	\addplot[const plot,cyan] file {wykresy/zadanie5_PID_V1_E1.txt};
	\end{axis}
	\end{tikzpicture}
	
	\caption{$e_1$ dla V1 (CO TO??)}
	\label{odpskokUdmc3}
\end{figure}

\begin{figure}[ht]
	\centering
	\begin{tikzpicture}
	\begin{axis}[
	width=7.0in,
	height=2.4in,
	xmin=0,xmax=1600,ymin=-2,ymax=2,
	xlabel={$k$},
	ylabel={$e_2$},
	legend pos=south east,
	y tick label style={/pgf/number format/1000 sep=},
	]
	\addplot[const plot,cyan] file {wykresy/zadanie5_PID_V1_E2.txt};
	\end{axis}
	\end{tikzpicture}
	
	\caption{$e_2$ dla V1 (CO TO??)}
	\label{odpskokUdmc3}
\end{figure}

%%%%%%%%%%%%%%%%%%%%%%%%%%%%%%%%%%%%%%%%%%%%%%%%%%%%%%%%%%%%%%%%%%%%%%V2

\begin{figure}[ht]
	\centering
	\begin{tikzpicture}
	\begin{axis}[
	width=7.0in,
	height=2.4in,
	xmin=0,xmax=1600,ymin=-1,ymax=4,
	xlabel={$k$},
	ylabel={$y_1$},
	legend pos=south east,
	y tick label style={/pgf/number format/1000 sep=},
	]
	\addplot[const plot,red, densely dashed] file {wykresy/zadanie5_PID_V1_Y1_zad.txt};
	\addplot[const plot,cyan] file {wykresy/zadanie5_PID_V2_Y1.txt};
	\legend{$y_1^{zad}$, $y_1$}
	\end{axis}
	\end{tikzpicture}
	
	\caption{Sygna� wyj�ciowy $y_1$ dla V2 (CO TO??)}
	\label{odpskokUdmc3}
\end{figure}

\begin{figure}[ht]
	\centering
	\begin{tikzpicture}
	\begin{axis}[
	width=7.0in,
	height=2.4in,
	xmin=0,xmax=1600,ymin=-1,ymax=4,
	xlabel={$k$},
	ylabel={$y_2$},
	legend pos=south east,
	y tick label style={/pgf/number format/1000 sep=},
	]
	\addplot[const plot,red, densely dashed] file {wykresy/zadanie5_PID_V1_Y2_zad.txt};
	\addplot[const plot,cyan] file {wykresy/zadanie5_PID_V2_Y2.txt};
	\legend{$y_2^{zad}$, $y_2$}
	\end{axis}
	\end{tikzpicture}
	
	\caption{Sygna� wyj�ciowy $y_1$ dla V2 (CO TO??)}
	\label{odpskokUdmc3}
\end{figure}

\begin{figure}[ht]
	\centering
	\begin{tikzpicture}
	\begin{axis}[
	width=7.0in,
	height=2.4in,
	xmin=0,xmax=1600,ymin=-5,ymax=5,
	xlabel={$k$},
	ylabel={$u_1$},
	legend pos=south east,
	y tick label style={/pgf/number format/1000 sep=},
	]
	\addplot[const plot,cyan] file {wykresy/zadanie5_PID_V2_U1.txt};
	\end{axis}
	\end{tikzpicture}
	
	\caption{Sygna� wej�ciowy $u_1$ dla V2 (CO TO??)}
	\label{odpskokUdmc3}
\end{figure}




\begin{figure}[ht]
	\centering
	\begin{tikzpicture}
	\begin{axis}[
	width=7.0in,
	height=2.4in,
	xmin=0,xmax=1600,ymin=-2,ymax=2,
	xlabel={$k$},
	ylabel={$u_2$},
	legend pos=south east,
	y tick label style={/pgf/number format/1000 sep=},
	]
	\addplot[const plot,cyan] file {wykresy/zadanie5_PID_V2_U2.txt};
	\end{axis}
	\end{tikzpicture}
	
	\caption{Sygna� wej�ciowy $u_2$ dla V2 (CO TO??)}
	\label{odpskokUdmc3}
\end{figure}

\begin{figure}[ht]
	\centering
	\begin{tikzpicture}
	\begin{axis}[
	width=7.0in,
	height=2.4in,
	xmin=0,xmax=1600,ymin=-2,ymax=2,
	xlabel={$k$},
	ylabel={$e_1$},
	legend pos=south east,
	y tick label style={/pgf/number format/1000 sep=},
	]
	\addplot[const plot,cyan] file {wykresy/zadanie5_PID_V2_E1.txt};
	\end{axis}
	\end{tikzpicture}
	
	\caption{$e_1$ dla V2 (CO TO??)}
	\label{odpskokUdmc3}
\end{figure}

\begin{figure}[ht]
	\centering
	\begin{tikzpicture}
	\begin{axis}[
	width=7.0in,
	height=2.4in,
	xmin=0,xmax=1600,ymin=-2,ymax=2,
	xlabel={$k$},
	ylabel={$e_2$},
	legend pos=south east,
	y tick label style={/pgf/number format/1000 sep=},
	]
	\addplot[const plot,cyan] file {wykresy/zadanie5_PID_V2_E2.txt};
	\end{axis}
	\end{tikzpicture}
	
	\caption{$e_2$ dla V2 (CO TO??)}
	\label{odpskokUdmc3}
\end{figure}r
	%PROJ4
\chapter{Punkt 6}

Rozmyty regulator DMC:\\\\
b�edy:  \\
1 regulator E = $\num{ 179,4611}$\\
2 regulatory E = $\num{148,1608}$\\
3 regulatory E = $\num{149,4896}$\\
4 regulatory E = $\num{147,9031}$\\
5 regulator�w E = $\num{134,2346}$\\\\

Podobnie jak w poprzednim przypadku, regulacja rozmyta daje znacznie lepsze efekty pod wzgl�dem jako�ciowym i ilo�ciowym; w ka�dym przypadku b��d E jest znacznie mniejszy w por�wnaniu do regulacji liniowej.
Ponownie, wachania b��du wzgl�dem regulator�w lokalnych wynikaj� z dobrania 'na oko'(eksperymentalnego) parametr�w funkcji przynale�no�ci oraz parametr�w $\lambda$, jednak tak jak si� mo�na spodziewa�, najlepszy wynik da� regulator z pi�cioma regulatorami lokalnymi.


\begin{figure}[h!]
	\centering
	
	\begin{tikzpicture}
	\begin{axis}[
	width=4.667in,
	height=1.8in,
	enlargelimits=false,
	%xmin=0,xmax=1600,
	ymin=-2.5,ymax=0.5,
	xlabel={$k$},
	ylabel={$Y$},
	legend pos=north east,
	y tick label style={/pgf/number format/1000 sep=},
	]
	\addplot[const plot,blue] file {../wykresy/zad6_y_1_reg.txt};
	\addplot[red] file {../wykresy/yzad5.txt};
	\legend{$Y$, $Y_{zad}$}
	\end{axis}
	\end{tikzpicture}
	
	\begin{tikzpicture}
	\begin{axis}[
	width=4.667in,
	height=1.8in,
	enlargelimits=false,
	%xmin=0,xmax=1600,
	ymin=-1.5,ymax=1.5,
	xlabel={$k$},
	ylabel={$U$},
	legend pos=north east,
	y tick label style={/pgf/number format/1000 sep=},
	]
	\addplot[const plot,blue] file {../wykresy/zad6_U_1_reg.txt};
	\end{axis}
	\end{tikzpicture}
	
	
	\caption{Przebiegi sygna��w dla 1 regulatora}
\end{figure}

\begin{figure}[h!]
	\centering
	
	\begin{tikzpicture}
	\begin{axis}[
	width=4.667in,
	height=1.8in,
	enlargelimits=false,
	%xmin=0,xmax=1600,
	ymin=-2.5,ymax=0.5,
	xlabel={$k$},
	ylabel={$Y$},
	legend pos=north east,
	y tick label style={/pgf/number format/1000 sep=},
	]
	\addplot[const plot,blue] file {../wykresy/zad6_y_2_reg.txt};
	\addplot[red] file {../wykresy/yzad5.txt};
	\legend{$Y$, $Y_{zad}$}
	\end{axis}
	\end{tikzpicture}
	
	\begin{tikzpicture}
	\begin{axis}[
	width=4.667in,
	height=1.8in,
	enlargelimits=false,
	%xmin=0,xmax=1600,
	ymin=-1.5,ymax=1.5,
	xlabel={$k$},
	ylabel={$U$},
	legend pos=north east,
	y tick label style={/pgf/number format/1000 sep=},
	]
	\addplot[const plot,blue] file {../wykresy/zad6_U_2_reg.txt};
	\end{axis}
	\end{tikzpicture}
	
	
	\caption{Przebiegi sygna��w dla 2 regulator�w}
\end{figure}

\begin{figure}[h!]
	\centering
	
	\begin{tikzpicture}
	\begin{axis}[
	width=4.667in,
	height=1.8in,
	enlargelimits=false,
	%xmin=0,xmax=1600,
	ymin=-2.5,ymax=0.5,
	xlabel={$k$},
	ylabel={$Y$},
	legend pos=north east,
	y tick label style={/pgf/number format/1000 sep=},
	]
	\addplot[const plot,blue] file {../wykresy/zad6_y_3_reg.txt};
	\addplot[red] file {../wykresy/yzad5.txt};
	\legend{$Y$, $Y_{zad}$}
	\end{axis}
	\end{tikzpicture}
	
	\begin{tikzpicture}
	\begin{axis}[
	width=4.667in,
	height=1.8in,
	enlargelimits=false,
	%xmin=0,xmax=1600,
	ymin=-1.5,ymax=1.5,
	xlabel={$k$},
	ylabel={$U$},
	legend pos=north east,
	y tick label style={/pgf/number format/1000 sep=},
	]
	\addplot[const plot,blue] file {../wykresy/zad6_U_3_reg.txt};
	\end{axis}
	\end{tikzpicture}
	
	
	\caption{Przebiegi sygna��w dla 3 regulator�w}
\end{figure}

\begin{figure}[h!]
	\centering
	
	\begin{tikzpicture}
	\begin{axis}[
	width=4.667in,
	height=1.8in,
	enlargelimits=false,
	%xmin=0,xmax=1600,
	ymin=-2.5,ymax=0.5,
	xlabel={$k$},
	ylabel={$Y$},
	legend pos=north east,
	y tick label style={/pgf/number format/1000 sep=},
	]
	\addplot[const plot,blue] file {../wykresy/zad6_y_4_reg.txt};
	\addplot[red] file {../wykresy/yzad5.txt};
	\legend{$Y$, $Y_{zad}$}
	\end{axis}
	\end{tikzpicture}
	
	\begin{tikzpicture}
	\begin{axis}[
	width=4.667in,
	height=1.8in,
	enlargelimits=false,
	%xmin=0,xmax=1600,
	ymin=-1.5,ymax=1.5,
	xlabel={$k$},
	ylabel={$U$},
	legend pos=north east,
	y tick label style={/pgf/number format/1000 sep=},
	]
	\addplot[const plot,blue] file {../wykresy/zad6_U_4_reg.txt};
	\end{axis}
	\end{tikzpicture}
	
	
	\caption{Przebiegi sygna��w dla 4 regulator�w}
\end{figure}

\begin{figure}[h!]
	\centering
	
	\begin{tikzpicture}
	\begin{axis}[
	width=4.667in,
	height=1.8in,
	enlargelimits=false,
	%xmin=0,xmax=1600,
	ymin=-2.5,ymax=0.5,
	xlabel={$k$},
	ylabel={$Y$},
	legend pos=north east,
	y tick label style={/pgf/number format/1000 sep=},
	]
	\addplot[const plot,blue] file {../wykresy/zad6_y_5_reg.txt};
	\addplot[red] file {../wykresy/yzad5.txt};
	\legend{$Y$, $Y_{zad}$}
	\end{axis}
	\end{tikzpicture}
	
	\begin{tikzpicture}
	\begin{axis}[
	width=4.667in,
	height=1.8in,
	enlargelimits=false,
	%xmin=0,xmax=1600,
	ymin=-1.5,ymax=1.5,
	xlabel={$k$},
	ylabel={$U$},
	legend pos=north east,
	y tick label style={/pgf/number format/1000 sep=},
	]
	\addplot[const plot,blue] file {../wykresy/zad6_U_5_reg.txt};
	\end{axis}
	\end{tikzpicture}
	
	
	\caption{Przebiegi sygna��w dla 5 regulator�w}
\end{figure}


%	%LAB4
\chapter{Punkt 7}


	Przebiegi regulacji DMC dla 2 regulator�w lokalnych r�wnie� s� dosy� zadowalaj�ce, sygna� wyj�ciowy zbiega do warto�ci zadanej. Przy zastosowaniu wi�kszej liczby regulator�w, oraz wykonaniu kliku eksperyment�w na rzeczywistym obiekcie w celu dobrania optymalnych nastaw prawdopodobnie uda�oby si� polepszy� rezultaty regulacji.


\begin{figure}[h!]
	\centering
	
	\begin{tikzpicture}
	\begin{axis}[
	width=4.667in,
	height=1.8in,
	enlargelimits=false,
	%xmin=0,xmax=1600,
	ymin=20,ymax=60,
	xlabel={$k$},
	ylabel={$Y$},
	legend pos=north east,
	y tick label style={/pgf/number format/1000 sep=},
	]
	\addplot[const plot,blue] file {wykresy/zadanie6_DMC_2reg_Y.txt};
	\addplot[red] file {wykresy/zadanie6_DMC_2reg_Yzad.txt};
	\legend{$Y$, $Y_{zad}$}
	\end{axis}
	\end{tikzpicture}
	
	\begin{tikzpicture}
	\begin{axis}[
	width=4.667in,
	height=1.8in,
	enlargelimits=false,
	%xmin=0,xmax=1600,
	ymin=0,ymax=100,
	xlabel={$k$},
	ylabel={$U$},
	legend pos=north east,
	y tick label style={/pgf/number format/1000 sep=},
	]
	\addplot[const plot,blue] file {wykresy/zadanie6_DMC_2reg_U.txt};
	\end{axis}
	\end{tikzpicture}
	
	\begin{tikzpicture}
	\begin{axis}[
	width=4.667in,
	height=1.8in,
	enlargelimits=false,
	%xmin=0,xmax=1600,
	ymin=-5,ymax=20,
	xlabel={$k$},
	ylabel={$E$},
	legend pos=north east,
	y tick label style={/pgf/number format/1000 sep=},
	]
	\addplot[const plot,blue] file {wykresy/zadanie6_DMC_2reg_blad.txt};
	\end{axis}
	\end{tikzpicture}
	
	\caption{Przebiegi sygna��w uzyskanych podczas eksperymentu na rzeczywistym obiekcie w przypadku zastosowania dw�ch regulator�w DMC}
\end{figure}
%	%LAB4
\chapter{Punkt 8}





	

\end{document}


