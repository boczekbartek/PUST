%LAB4
\chapter{Punkt 5}


 Podczas labolatorium uda�o nam si� wykona� symulacj� jedynie dla przypadku z pomiarem zak��cenia, jednak nasze teorytyczne do�wiadczenia na projekcie pokaza�y, �e pomiar zak��cenia znacznie poprawia jako�� regulacji, a co za tym idzie mo�emy si� spodziewa�, �e uzyskany przez nas wynik jest lepszy ni� ten, kt�ry uzyskaliby�my bez pomiaru zak��cenia. Nastawy regulatora DMC jakich u�yli�my podczas symulacji: $D=\num{110}$; $N=\num{130}$; $N_u=\num{6}$; $\lambda=\num{2}$; $D_z=\num{50}$;


\begin{figure}[h!]
	\centering
	
	\begin{tikzpicture}
	\begin{axis}[
	width=4.667in,
	height=1.2in,
	xmin=0,xmax=700,ymin=31,ymax=42,
	xlabel={$k$},
	ylabel={$Y$},
	legend pos=south east,
	y tick label style={/pgf/number format/1000 sep=},
	]
	\addplot[const plot,blue] file {wykresy/zadanie5pomiarY.txt};
	\addplot[const plot,red, densely dashed] file {wykresy/zadanie5pomiarYzad.txt};
	\legend{$Y$,$Y^{zad}$}
	\end{axis}
	\end{tikzpicture}
	
	\begin{tikzpicture}
	\begin{axis}[
	width=4.667in,
	height=1.2in,
	xmin=0,xmax=700,ymin=3,ymax=74,
	xlabel={$k$},
	ylabel={$U$},
	legend pos=south east,
	y tick label style={/pgf/number format/1000 sep=},
	]
	\addplot[const plot,blue] file {wykresy/zadanie5pomiarU.txt};
	\end{axis}
	\end{tikzpicture}
	\begin{tikzpicture}
	\begin{axis}[
	width=4.667in,
	height=1.1in,
	xmin=0,xmax=700,ymin=0,ymax=40,
	xlabel={$k$},
	ylabel={$Z$},
	legend pos=south east,
	y tick label style={/pgf/number format/1000 sep=},
	]
	\addplot[const plot,blue] file {wykresy/zadanie5pomiarZ.txt};
	\end{axis}
	\end{tikzpicture}
	
	
	\caption{Przebiegi sygna��w regulatora DMC przy skokowej zmianie sygna�u zak��caj�cego}
	\label{bezPomZak}
\end{figure}
\FloatBarrier


