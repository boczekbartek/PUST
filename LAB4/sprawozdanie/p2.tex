%LAB4
\chapter{Punkt 2}


\begin{figure}[ht]
	\centering
		
	\begin{tikzpicture}
	\begin{axis}[
	enlargelimits=false,
	width=4.667in,
	height=5in,
	%xmin=0,xmax=350,ymin=31,ymax=36,
	xlabel={$k$},
	ylabel={$Y$},
	legend pos=south east,
	y tick label style={/pgf/number format/1000 sep=},
	]
	\addplot[const plot,green] file {wykresy/zadanie2_skok40.txt};
	\addplot[const plot,red] file {wykresy/przykladowe/zadanie2_Z20.txt};
	\addplot[const plot,blue] file {wykresy/przykladowe/zadanie2_Z30.txt};
	\addplot[const plot,magenta] file {wykresy/przykladowe/zadanie2_Z30.txt};
	\addplot[const plot,cyan] file {wykresy/przykladowe/zadanie2_Z30.txt};
	\legend{$G1_{\mathrm{skok}}=\num{40}$,$G1_{\mathrm{skok}}=\num{45}$,$G1_{\mathrm{skok}}=\num{50}$,$G1_{\mathrm{skok}}=\num{55}$,$G1_{\mathrm{skok}}=\num{50}$}
	\end{axis}
	\end{tikzpicture}
	
	\caption{Odpowiedzi skokowe procesu dla pi�ciu r�nych zmian sygna�u steruj�cego}
	\label{odpWeWysef}
\end{figure}
