%LAB4
\chapter{Punkt 4}



Zaimplementowali�my algorytmy do regulacji PID i DMC.
W tym punkcie warto doda�, �e przed przyst�pieniem do eksperyment�w na rzeczywistym obiekcie stworzyli�my jego model na podstawie jego odpowiedzi skokowych. 
Dzi�ki temu mogli�my go symulowa� w �rodowisku Matlab i tutaj te� ustali� optymalne parametry algorytm�w regulacji.\\

Do cel�w symulacyjnych przygotowali�my dwa modele naszego obiektu - pierwszy wykorzystuj�c otrzyman� w zadaniu 3 aproksymacj� odpowiedzi skokowej dla pierwszej z liniowych cz�sci charakterystyki statycznej (czyli dla  $\num{0} \leq U \leq \num{50}$) oraz drugi na drugiej liniowej cze�ci tej�e charakterystyki ( $\num{50} \leq U \leq \num{100}$).\\\\

Nast�pnie wyznaczyli�my korzystaj�c z optymalizatora
 \verb|ga|
(ograniczenia lb=[$\num{0,01}$;$\num{1}$;$\num{0,01}$]; 
ub = [$\num{50}$;$\num{50}$;$\num{50}$]) a nast�nie eksperymentalnie dostrajaj�c parametry algorytm�w PID i DMC ka�dego z liniowych przedzia��w charakterystyki statycznej. Otrzymane rezultaty:\\\\

dla przedzia�u  $\num{0} \leq U \leq \num{50}$:\\
PID:\\
K=$\num{0,14661}$; $T_i$=$\num{1,63138}$; $T_d$=$\num{0,010002}$;\\
DMC:\\
N=$\num{0}$; $N_u$=$\num{0}$; $\lambda=\num{0}$\\\\


dla przedzia�u $\num{50} \leq U \leq \num{100}$:\\
PID:\\
K=$\num{0,194661}$; $T_i$=$\num{1,53138}$; $T_d$=$\num{0,010002}$;\\
DMC:\\
N=$\num{0}$; $N_u$=$\num{0}$; $\lambda$=$\num{0}$\\\\

!!!!!!!!!!!!!!!!!!!!!!!!!!!!!!!!!!!!!!!!!!!!!!TODO - podmien wartosci DMC
