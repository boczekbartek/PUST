%LAB2
\chapter{Punkt 2}

W�a�ciwo�ci statyczne obiektu mo�na uzna� za w przybli�eniu liniowe, gdy� mo�emy zauwa�y�, �e skoki wyj�cia s� proporcjonalne do skok�w zak��cenia. (skok Y dla $Z_{skok}=\num{10}$ jest ok 2 razy mniejszy ni� skok Y dla $Z_{skok}=\num{20}$). Mo�emy zatem obliczy� wzmocnienie statyczne, kt�re jest r�wne: $K_{stat}=0,1465$.

\begin{figure}[ht]
	\centering
		
	\begin{tikzpicture}
	\begin{axis}[
	width=4.667in,
	height=5in,
	xmin=0,xmax=350,ymin=31,ymax=36,
	xlabel={$k$},
	ylabel={$Y$},
	legend pos=south east,
	y tick label style={/pgf/number format/1000 sep=},
	]
	\addplot[const plot,green] file {wykresy/zadanie2_Z10.txt};
	\addplot[const plot,red] file {wykresy/zadanie2_Z20.txt};
	\addplot[const plot,blue] file {wykresy/zadanie2_Z30.txt};
	\legend{$Z_{\mathrm{skok}}=\num{10}$,$Z_{\mathrm{skok}}=\num{20}$,$Z_{\mathrm{skok}}=\num{30}$}
	\end{axis}
	\end{tikzpicture}
	
	\caption{Odpowiedzi skokowe toru zak��cenie-wyj�cie procesu dla trzech r�nych zmian sygna�u zak��caj�cego}
	\label{odpWeWy}
\end{figure}
