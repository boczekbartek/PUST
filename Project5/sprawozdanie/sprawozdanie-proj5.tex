%LAB4
\documentclass[a4paper,titlepage,11pt,twosides,floatssmall]{mwrep}
\usepackage[left=2.5cm,right=2.5cm,top=2.5cm,bottom=2.5cm]{geometry}
\usepackage[OT1]{fontenc}
\usepackage{polski}
\usepackage{amsmath}
\usepackage{amsfonts}
\usepackage{amssymb}
\usepackage{graphicx}
\usepackage{url}
\usepackage{tikz}
\usetikzlibrary{arrows,calc,decorations.markings,math,arrows.meta}
\usepackage{rotating}
\usepackage[percent]{overpic}
\usepackage[cp1250]{inputenc}
\usepackage{xcolor}
\usepackage{pgfplots}
\usetikzlibrary{pgfplots.groupplots}
\usepackage{listings}
\usepackage{matlab-prettifier}
\usepackage{siunitx}
\usepackage{placeins}
\usepackage{gensymb}
\usepackage{verbatim}
\definecolor{szary}{rgb}{0.95,0.95,0.95}
\sisetup{detect-weight,exponent-product=\cdot,output-decimal-marker={,},per-mode=symbol,binary-units=true,range-phrase={-},range-units=single}
\SendSettingsToPgf

%konfiguracje pakietu listings
\lstset{
	backgroundcolor=\color{szary},
	frame=single,
	breaklines=true,
}
\lstdefinestyle{customlatex}{
	basicstyle=\footnotesize\ttfamily,
	%basicstyle=\small\ttfamily,
}
\lstdefinestyle{customc}{
	breaklines=true,
	frame=tb,
	language=C,
	xleftmargin=0pt,
	showstringspaces=false,
	basicstyle=\small\ttfamily,
	keywordstyle=\bfseries\color{green!40!black},
	commentstyle=\itshape\color{purple!40!black},
	identifierstyle=\color{blue},
	stringstyle=\color{orange},
}
\lstdefinestyle{custommatlab}{
	captionpos=t,
	breaklines=true,
	frame=tb,
	xleftmargin=0pt,
	language=matlab,
	showstringspaces=false,
	%basicstyle=\footnotesize\ttfamily,
	basicstyle=\scriptsize\ttfamily,
	keywordstyle=\bfseries\color{green!40!black},
	commentstyle=\itshape\color{purple!40!black},
	identifierstyle=\color{blue},
	stringstyle=\color{orange},
}

%wymiar tekstu (bez �ywej paginy)
\textwidth 160mm \textheight 247mm

%ustawienia pakietu pgfplots
\pgfplotsset{
	tick label style={font=\scriptsize},
	label style={font=\small},
	legend style={font=\small},
	title style={font=\small}
}

\def\figurename{Rys.}
\def\tablename{Tab.}

%konfiguracja liczby p�ywaj�cych element�w
\setcounter{topnumber}{0}%2
\setcounter{bottomnumber}{3}%1
\setcounter{totalnumber}{5}%3
\renewcommand{\textfraction}{0.01}%0.2
\renewcommand{\topfraction}{0.95}%0.7
\renewcommand{\bottomfraction}{0.95}%0.3
\renewcommand{\floatpagefraction}{0.35}%0.5

\begin{document}
	\frenchspacing
	\pagestyle{uheadings}
	
	%strona tytu�owa
	\title{\bf Sprawozdanie z projektu nr 5\vskip 0.1cm}
	\author{Bart�omiej Boczek, Aleksander Piotrowski, �ukasz �migielski}
	\date{11 czerwca 2017}
	
	\makeatletter
	\renewcommand{\maketitle}{\begin{titlepage}
			\begin{center}{\LARGE {\bf
						Wydzia� Elektroniki i Technik Informacyjnych}}\\
				\vspace{0.4cm}
				{\LARGE {\bf Politechnika Warszawska}}\\
				\vspace{0.3cm}
			\end{center}
			\vspace{5cm}
			\begin{center}
				{\bf \LARGE Projektowanie uk�ad�w sterowania\\ (projekt grupowy) \vskip 0.1cm}
			\end{center}
			\vspace{1cm}
			\begin{center}
				{\bf \LARGE \@title}
			\end{center}
			\vspace{2cm}
			\begin{center}
				{\bf \Large \@author \par}
			\end{center}
			\vspace*{\stretch{6}}
			\begin{center}
				\bf{\large{Warszawa, \@date\vskip 0.1cm}}
			\end{center}
		\end{titlepage}
	}
	\makeatother
	
	\maketitle
	
	\tableofcontents

	%LAB3
\chapter{Punkt 1 }

Komunikacja z obiektem grzej�co-ch�odz�cym przebiega poprzez port szeregowy. Uda�o nam si� skomunikowa� ze �rodowiskiem za pomoc� funkcji \verb|readMeasurements| oraz \verb|sendControls| dostarczonymi w pakiecie laboratoryjnym. Wysy�anie sygna��w steruj�cych do wiatraczka, grza�ki oraz odczyt pomiar�w z czujnik�w znajduj�cych si� na obiekcie
przebiega� pomy�lnie. Warto�ci, kt�rymi b�dziemy sterowa� w tym �wiczeniu to W1 - 1 paramert funkcji \verb|sendControls|, oraz G1 - 5 parametr. Pomiarem nas interesuj�cym b�dzie pierwsza warto��
w wektorze zwracanym przez \verb|readMeasurements|, czyli pomiar temperatury T1.

Wyznaczanie punktu pracy:\\
wys�ali�my sta�� warto�� sterowania $U_{pp} = \num{35}\%$ oraz zmierzyli�my warto�� wyj�cia. Wiatraczek W1 (cecha �rodowiska) zosta� ustawiony na $W1 = \num{50}\%$ obrot�w maksymalnych.\\

Poni�ej przedstawiony zosta� przebieg eksperymentu:

\begin{figure}[ht]
	\centering
	\begin{tikzpicture}
	\begin{axis}[
	width=5.0in,
	height=3.4in,
	xmin=0,xmax=300,ymin=32,ymax=34,
	xlabel={$k$},
	ylabel={$T1$},
	legend pos=south east,
	y tick label style={/pgf/number format/1000 sep=},
	]
	\addplot[const plot,cyan] file {wykresy/zadanie1.txt};
	\end{axis}
	\end{tikzpicture}

	\caption{Punkt pracy}
	
\end{figure}

\begin{figure}
	Na wykresie mo�emy zaobserwowa�, �e warto�� temperatury T1 (wyj�cia obiektu) stabilizuje sie na warto�ci $\num{33,25}\degree$C.
	Zatem mo�emy stwierdzi�, �e punkt pracy obiektu to: ($U_{pp}$; $Y_{pp}$) = ($\num{35}$; $\num{33,25}$).
	
\end{figure}
	%LAB4
\chapter{Punkt 2}


Zebrane symulacyjnie odpowiedzi skokowe ('r�wnoleg�e' i skro�ne) dla skok�w z punkt�w pracy $U_{pp}$=0 do U=1 w chwili k=0.

\begin{figure}
	\centering
		
	\begin{tikzpicture}
	\begin{axis}[
	enlargelimits=false,
	width=4.667in,
	height=4in,
	xmin=10,xmax=200,
	ymin=-0.1,ymax=2.0,
	xlabel={$k$},
	ylabel={$Y$},
	legend pos=south east,
	y tick label style={/pgf/number format/1000 sep=},
	]
	\addplot[const plot,green] file {wykresy/zad2u1y1.txt};
	\addplot[const plot,yellow] file {wykresy/zad2u1y2.txt};
	\addplot[const plot,gray] file {wykresy/zad2u1y3.txt};
	\legend{$Y1$,
			$Y2$,
			$Y3$,
	}
	\end{axis}
	\end{tikzpicture}
	\caption{Przy skoku U1}
\end{figure}

\begin{figure}
	\centering
	\begin{tikzpicture}
	\begin{axis}[
	enlargelimits=false,
	width=4.667in,
	height=4in,
	xmin=10,xmax=200,
	ymin=-0.1,ymax=2.0,
	xlabel={$k$},
	ylabel={$Y$},
	legend pos=south east,
	y tick label style={/pgf/number format/1000 sep=},
	]
	\addplot[const plot,green] file {wykresy/zad2u2y1.txt};
	\addplot[const plot,yellow] file {wykresy/zad2u2y2.txt};
	\addplot[const plot,gray] file {wykresy/zad2u2y3.txt};
	\legend{$Y1$,
		$Y2$,
		$Y3$,
	}
	\end{axis}
	\end{tikzpicture}
	\caption{Przy skoku U2}
\end{figure}

\begin{figure}
	\centering
	\begin{tikzpicture}
	\begin{axis}[
	enlargelimits=false,
	width=4.667in,
	height=4in,
	xmin=10,xmax=200,
	ymin=-0.1,ymax=2.0,
	xlabel={$k$},
	ylabel={$Y$},
	legend pos=south east,
	y tick label style={/pgf/number format/1000 sep=},
	]
	\addplot[const plot,green] file {wykresy/zad2u3y1.txt};
	\addplot[const plot,yellow] file {wykresy/zad2u3y2.txt};
	\addplot[const plot,gray] file {wykresy/zad2u3y3.txt};
	\legend{$Y1$,
		$Y2$,
		$Y3$,
	}
	\end{axis}
	\end{tikzpicture}
	\caption{Przy skoku U3}
\end{figure}

\begin{figure}
	\centering
	\begin{tikzpicture}
	\begin{axis}[
	enlargelimits=false,
	width=4.667in,
	height=4in,
	xmin=10,xmax=200,
	ymin=-0.1,ymax=2.0,
	xlabel={$k$},
	ylabel={$Y$},
	legend pos=south east,
	y tick label style={/pgf/number format/1000 sep=},
	]
	\addplot[const plot,green] file {wykresy/zad2u4y1.txt};
	\addplot[const plot,yellow] file {wykresy/zad2u4y2.txt};
	\addplot[const plot,gray] file {wykresy/zad2u4y3.txt};
	\legend{$Y1$,
		$Y2$,
		$Y3$,
	}
	\end{axis}
	\end{tikzpicture}
	\caption{Przy skoku U4}
	
\end{figure}



	%LAB4
\chapter{Punkt 3}

Obliczyli�my dwie odpowiedzi skokowe, odpowiednio dla dw�ch skok�w odpowiadaj�cym dw�m liniowym cz�sciom charakterystyki statycznej. Pierwszy to skok sterowania U z $\num{35}\% \rightarrow \num{50}\%$ (gdy� w�a�nie w punkcie $\num{50}$ jest przegi�cie charakterystyki statycznej), drugi z $\num{35}\% \rightarrow \num{90}\%$.
Skoki zosta�y przeskalowane, tak aby mog�y by� uznane za odpowied� skokow�, czyli odpowied� obiektu na skok jednostkowy.\\ 

Nast�pnym krokiem by�a aproksymacja uzyskanych wcze�niej odpowiedzi skokowych cz�onem inercyjnym drugiego rz�du z op�nieniem.
Parametry T1, T2 oraz K tego cz�ony wyznaczone zosta�y za pomoc� optymalizatora \verb|ga|. Jako, �e jest on optymalizatorem niedeterministycznym (nie daje przy ka�dym uruchomieniu takich samych rezultat�w),
kilka razy powtarzali�my nasze eksperymenty oraz oceniali�my wizualnie (na wykresach) jako�� dopasowania funkcji aproksymuj�cej do obliczonej odpowiedzi skokowej.
Okaza�o si�, �e bardzo dobre rezultaty udawa�o si� uzyska� nie podaj�c optymalizatorowi �adnych ogranicze�, dlatego te� nie zdecydowali�my si� na tego typu kroki.\\\\

Parametry cz�onu inercyjnego drugiego rz�du z op�nieniem wyznaczone przy pomocy optymalizatora:
dla odpowiedzi skokowej obliczonej dla skoku sterowania do warto�ci $\num{50}\%$:\\
$T1=\num{14,875560};\\    T2=\num{77,388071};\\    K=\num{0,461244}$;\\\\

dla odpowiedzi skokowej obliczonej dla skoku sterowania do warto�ci $\num{90}\%$:\\
$T1=\num{13,342281};\\    T2=\num{76,79432};\\    K=\num{0,286991}$;\\\\

Rezultaty, kt�re uda�o nam si� uzyska� zosta�y przedstawione na poni�szych wykresach:

\begin{figure}[ht]
	\centering
	\begin{tikzpicture}
	\begin{axis}[
	enlargelimits=false,
	width=5.0in,
	height=3.4in,
	%xmin=0,xmax=350,
	ymin=-0.05,ymax=0.5,
	xlabel={$k$},
	ylabel={$Y$},
	legend pos=south east,
	y tick label style={/pgf/number format/1000 sep=},
	]
	\addplot[const plot,red] file {wykresy/odp_skokowa_skok_50.txt};
	\addplot[cyan] file {wykresy/funkcja_aproksymujaca_skok_50.txt};
	\legend{orygina�,aproksymacja}
	\end{axis}
	\end{tikzpicture}
	\caption{Por�wnanie odpowiedzi skokowej i jej aproksymacji przy skoku sterowania do warto�ci 50\%}
\end{figure}


\begin{figure}[ht]
	\centering
	\begin{tikzpicture}
	\begin{axis}[
	enlargelimits=false,
	width=5.0in,
	height=3.4in,
	%xmin=0,xmax=350,
	ymin=-0.05,ymax=0.5,
	xlabel={$k$},
	ylabel={$Y$},
	legend pos=south east,
	y tick label style={/pgf/number format/1000 sep=},
	]
	\addplot[const plot,red] file {wykresy/odp_skokowa_skok_90.txt};
	\addplot[cyan] file {wykresy/funkcja_aproksymujaca_skok_90.txt};
	\legend{orygina�,aproksymacja}
	\end{axis}
	\end{tikzpicture}
	\caption{Por�wnanie odpowiedzi skokowej i jej aproksymacji przy skoku sterowania do warto�ci 90\%}
\end{figure}

%	%LAB4
\chapter{Punkt 4}



Zaimplementowali�my algorytmy do regulacji PID i DMC.
W tym punkcie warto doda�, �e przed przyst�pieniem do eksperyment�w na rzeczywistym obiekcie stworzyli�my jego model na podstawie jego odpowiedzi skokowych. 
Dzi�ki temu mogli�my go symulowa� w �rodowisku Matlab i tutaj te� ustali� optymalne parametry algorytm�w regulacji.\\

Do cel�w symulacyjnych przygotowali�my dwa modele naszego obiektu - pierwszy wykorzystuj�c otrzyman� w zadaniu 3 aproksymacj� odpowiedzi skokowej dla pierwszej z liniowych cz�sci charakterystyki statycznej (czyli dla  $\num{0} \leq U \leq \num{50}$) oraz drugi na drugiej liniowej cze�ci tej�e charakterystyki ( $\num{50} \leq U \leq \num{100}$).\\\\

Nast�pnie wyznaczyli�my korzystaj�c z optymalizatora
 \verb|ga|
(ograniczenia lb=[$\num{0,01}$;$\num{1}$;$\num{0,01}$]; 
ub = [$\num{50}$;$\num{50}$;$\num{50}$]) a nast�pnie eksperymentalnie dostrajaj�c parametry algorytm�w PID i DMC ka�dego z liniowych przedzia��w charakterystyki statycznej. Otrzymane rezultaty:\\\\

dla przedzia�u  $\num{0} \leq U \leq \num{50}$:\\
PID:\\
K=$\num{0,14661}$; $T_i$=$\num{1,63138}$; $T_d$=$\num{0,010002}$;\\
DMC:\\
D=$\num{300}$; N=$\num{30}$; $N_u$=$\num{10}$; $\lambda=\num{2,4}$\\\\


dla przedzia�u $\num{50} \leq U \leq \num{100}$:\\
PID:\\
K=$\num{0,194661}$; $T_i$=$\num{1,53138}$; $T_d$=$\num{0,010002}$;\\
DMC:\\
D=$\num{300}$; N=$\num{40}$; $N_u$=$\num{32}$; $\lambda$=$\num{6,86402}$\\\\

Jako�� regulacji oceniali�my na podstawie symulacji przebieg�w regulacji z wykorzystaniem konkretnych modeli dla kilku skok�w warto�ci zadanej. Oto otrzymane rezultaty:\\\\


\begin{figure}[h!]
	\centering
	
	\begin{tikzpicture}
	\begin{axis}[
	width=4.667in,
	height=1.8in,
	enlargelimits=false,
	%xmin=0,xmax=1600,
	ymin=30,ymax=42,
	xlabel={$k$},
	ylabel={$Y$},
	legend pos=north east,
	y tick label style={/pgf/number format/1000 sep=},
	]
	\addplot[const plot,blue] file {wykresy/zadanie4_PID1_Y.txt};
	\addplot[red] file {wykresy/zadanie4_PID1_Yzad.txt};
	\legend{$Y$, $Y_{zad}$}
	\end{axis}
	\end{tikzpicture}
	
	\begin{tikzpicture}
	\begin{axis}[
	width=4.667in,
	height=1.8in,
	enlargelimits=false,
	%xmin=0,xmax=1600,
	ymin=34,ymax=46,
	xlabel={$k$},
	ylabel={$U$},
	legend pos=north east,
	y tick label style={/pgf/number format/1000 sep=},
	]
	\addplot[const plot,blue] file {wykresy/zadanie4_PID1_U.txt};
	\end{axis}
	\end{tikzpicture}
	
	\caption{Przebiegi symulacji dla regulatora PID dla pierwszej liniowej cz�ci charakterystyki statycznej}
\end{figure}


\begin{figure}[h!]
	\centering
	
	\begin{tikzpicture}
	\begin{axis}[
	width=4.667in,
	height=1.8in,
	enlargelimits=false,
	%xmin=0,xmax=1600,
	ymin=30,ymax=42,
	xlabel={$k$},
	ylabel={$Y$},
	legend pos=north east,
	y tick label style={/pgf/number format/1000 sep=},
	]
	\addplot[const plot,blue] file {wykresy/zadanie4_PID2_Y.txt};
	\addplot[red] file {wykresy/zadanie4_PID2_Yzad.txt};
	\legend{$Y$, $Y_{zad}$}
	\end{axis}
	\end{tikzpicture}
	
	\begin{tikzpicture}
	\begin{axis}[
	width=4.667in,
	height=1.8in,
	enlargelimits=false,
	%xmin=0,xmax=1600,
	ymin=34,ymax=46,
	xlabel={$k$},
	ylabel={$U$},
	legend pos=north east,
	y tick label style={/pgf/number format/1000 sep=},
	]
	\addplot[const plot,blue] file {wykresy/zadanie4_PID2_U.txt};
	\end{axis}
	\end{tikzpicture}
	
	\caption{Przebiegi symulacji dla regulatora PID dla drugiej liniowej cz�ci charakterystyki statycznej}
\end{figure}

%%%%%%%%%%%%%%%%%%%%%%%%%%%%%%%%%%%%%%%%%%%%%%%%%%%%%%%%%%%%%%%%%%%%%%%%%%%%%DMC

\begin{figure}[h!]
	\centering
	
	\begin{tikzpicture}
	\begin{axis}[
	width=4.667in,
	height=1.8in,
	enlargelimits=false,
	%xmin=0,xmax=1600,
	ymin=30,ymax=42,
	xlabel={$k$},
	ylabel={$Y$},
	legend pos=north east,
	y tick label style={/pgf/number format/1000 sep=},
	]
	\addplot[const plot,blue] file {wykresy/zadanie4_DMC1_Y.txt};
	\addplot[red] file {wykresy/zadanie4_DMC1_Yzad.txt};
	\legend{$Y$, $Y_{zad}$}
	\end{axis}
	\end{tikzpicture}
	
	\begin{tikzpicture}
	\begin{axis}[
	width=4.667in,
	height=1.8in,
	enlargelimits=false,
	xmin=0,xmax=1000,
	ymin=30,ymax=42,
	xlabel={$k$},
	ylabel={$U$},
	legend pos=north east,
	y tick label style={/pgf/number format/1000 sep=},
	]
	\addplot[const plot,blue] file {wykresy/zadanie4_DMC1_U.txt};
	\end{axis}
	\end{tikzpicture}
	
	\caption{Przebiegi symulacji dla regulatora DMC dla pierwszej liniowej cz�ci charakterystyki statycznej}
\end{figure}


\begin{figure}[h!]
	\centering
	
	\begin{tikzpicture}
	\begin{axis}[
	width=4.667in,
	height=1.8in,
	enlargelimits=false,
	%xmin=0,xmax=1600,
	ymin=30,ymax=42,
	xlabel={$k$},
	ylabel={$Y$},
	legend pos=north east,
	y tick label style={/pgf/number format/1000 sep=},
	]
	\addplot[const plot,blue] file {wykresy/zadanie4_DMC2_Y.txt};
	\addplot[red] file {wykresy/zadanie4_DMC2_Yzad.txt};
	\legend{$Y$, $Y_{zad}$}
	\end{axis}
	\end{tikzpicture}
	
	\begin{tikzpicture}
	\begin{axis}[
	width=4.667in,
	height=1.8in,
	enlargelimits=false,
	xmin=0,xmax=1000,
	ymin=30,ymax=42,
	xlabel={$k$},
	ylabel={$U$},
	legend pos=north east,
	y tick label style={/pgf/number format/1000 sep=},
	]
	\addplot[const plot,blue] file {wykresy/zadanie4_DMC2_U.txt};
	\legend{$U$}
	\end{axis}
	\end{tikzpicture}
	
	\caption{Przebiegi symulacji dla regulatora DMC dla drugiej liniowej cz�ci charakterystyki statycznej}
\end{figure}


%	%LAB4
\chapter{Punkt 5}

W tym zadaniu wykonali�my eksperyment z uzyciem pierwszego regulatora PID obliczonego w zadaniu poprzednim. Niestety, w zwi�zku z nieliniowo�ci� obiektu, regulacja nie daje zadowalaj�cych rezultat�w. Regulator PID to regulator liniowy, kt�ry mo�na zastosowa� jedynie do 
obiekt�w o charakterystyce liniowej.\\

W zwi�zku z ograniczeniem czasowym zaj�� labolatoryjnych oraz d�ugim czasem stabilizacji obiektu po skoku sterowania (oko�o 300 sekund), zdecydowali�my si� zastosowa� mniej skok�w sterowania ni� zalecane by�o w poleceniu.  

B��d skumulowany w tym przypadku wyni�s�: $\num{3762,4}$\\\\


\begin{figure}[h!]
	\centering
	
	\begin{tikzpicture}
	\begin{axis}[
	width=4.667in,
	height=1.8in,
	enlargelimits=false,
	%xmin=0,xmax=1600,
	ymin=25,ymax=55,
	xlabel={$k$},
	ylabel={$Y$},
	legend pos=north east,
	y tick label style={/pgf/number format/1000 sep=},
	]
	\addplot[const plot,blue] file {wykresy/zadanie5_PID_Y.txt};
	\addplot[red] file {wykresy/zadanie5_PID_Yzad.txt};
	\legend{$Y$, $Y_{zad}$}
	\end{axis}
	\end{tikzpicture}
	
	\begin{tikzpicture}
	\begin{axis}[
	width=4.667in,
	height=1.8in,
	enlargelimits=false,
	%xmin=0,xmax=1600,
	ymin=-5,ymax=105,
	xlabel={$k$},
	ylabel={$U$},
	legend pos=north east,
	y tick label style={/pgf/number format/1000 sep=},
	]
	\addplot[const plot,blue] file {wykresy/zadanie5_PID_U.txt};
	\end{axis}
	\end{tikzpicture}
	
		\begin{tikzpicture}
	\begin{axis}[
	width=4.667in,
	height=1.8in,
	enlargelimits=false,
	xmin=0,xmax=1050,
	ymin=-5,ymax=20,
	xlabel={$k$},
	ylabel={$E$},
	legend pos=north east,
	y tick label style={/pgf/number format/1000 sep=},
	]
	\addplot[const plot,blue] file {wykresy/zadanie5_PID_blad.txt};
	\end{axis}
	\end{tikzpicture}
	
	\caption{Przebiegi sygna��w uzyskanych podczas eksperymentu na rzeczywistym obiekcie w przypadku zastosowania 1 regulatora PID}
\end{figure}

	%LAB4
\chapter{Punkt 6}




\begin{figure}
	\centering
	
	\begin{tikzpicture}
	\begin{axis}[
	enlargelimits=false,
	width=4.667in,
	height=2in,
	%xmin=0,xmax=350
	ymin=-1,ymax=11,
	xlabel={$k$},
	ylabel={$Y1$},
	legend pos=south east,
	]
	\addplot[const plot,blue] file {wykresy/zad6/zad6_Y1.txt};
	\addplot[const plot,red] file {wykresy/zad6/zad6_Yzad1.txt};
	\legend{$Y1$,
		$Y1_{zad}$,}
	\end{axis}
	\end{tikzpicture}
	
	\begin{tikzpicture}
	\begin{axis}[
	enlargelimits=false,
	width=4.667in,
	height=1.5in,
	%xmin=0,xmax=350
	%ymin=-0.1,ymax=2.0,
	xlabel={$k$},
	ylabel={$U1$},
	legend pos=south east,
	]
	\addplot[const plot,blue] file {wykresy/zad6/zad6dmc_U1.txt};
	\end{axis}
	\end{tikzpicture}
	
	
	
	\begin{tikzpicture}
	\begin{axis}[
	enlargelimits=false,
	width=4.667in,
	height=2in,
	%xmin=0,xmax=350
	ymin=-1,ymax=11,
	xlabel={$k$},
	ylabel={$Y2$},
	legend pos=south east,
	]
	\addplot[const plot,blue] file {wykresy/zad6/zad6_Y2.txt};
	\addplot[const plot,red] file {wykresy/zad6/zad6_Yzad2.txt};
	\legend{$Y2$,
		$Y2_{zad}$,}
	\end{axis}
	\end{tikzpicture}
	\begin{tikzpicture}
	\begin{axis}[
	enlargelimits=false,
	width=4.667in,
	height=1.5in,
	%xmin=0,xmax=350
	%ymin=-0.1,ymax=2.0,
	xlabel={$k$},
	ylabel={$U2$},
	legend pos=south east,
	]
	\addplot[const plot,blue] file {wykresy/zad6/zad6dmc_U2.txt};
	\end{axis}
	\end{tikzpicture}
	
	
	
	\begin{tikzpicture}
	\begin{axis}[
	enlargelimits=false,
	width=4.667in,
	height=2in,
	%xmin=0,xmax=350
	%ymin=-0.1,ymax=2.0,
	ymin=-1,
	xlabel={$k$},
	ylabel={$Y3$},
	legend pos=south east,
	]
	\addplot[const plot,blue] file {wykresy/zad6/zad6_Y3.txt};
	\addplot[const plot,red] file {wykresy/zad6/zad6_Yzad3.txt};
	\legend{$Y3$,
		$Y3_{zad}$,}
	\end{axis}
	\end{tikzpicture}
	\begin{tikzpicture}
	\begin{axis}[
	enlargelimits=false,
	width=4.667in,
	height=1.5in,
	%xmin=0,xmax=350
	%ymin=-0.1,ymax=2.0,
	xlabel={$k$},
	ylabel={$U3$},
	legend pos=south east,
	]
	\addplot[const plot,blue] file {wykresy/zad6/zad6dmc_U3.txt};
	\end{axis}
	\end{tikzpicture}
	
	\caption{DMC - Sygna�y procesu, b��d E = $\num{375,535682}$}
\end{figure}

%	%LAB4
\chapter{Punkt 7}


	Przebiegi regulacji DMC dla 2 regulator�w lokalnych r�wnie� s� dosy� zadowalaj�ce, sygna� wyj�ciowy zbiega do warto�ci zadanej. Przy zastosowaniu wi�kszej liczby regulator�w, oraz wykonaniu kliku eksperyment�w na rzeczywistym obiekcie w celu dobrania optymalnych nastaw prawdopodobnie uda�oby si� polepszy� rezultaty regulacji.


\begin{figure}[h!]
	\centering
	
	\begin{tikzpicture}
	\begin{axis}[
	width=4.667in,
	height=1.8in,
	enlargelimits=false,
	%xmin=0,xmax=1600,
	ymin=20,ymax=60,
	xlabel={$k$},
	ylabel={$Y$},
	legend pos=north east,
	y tick label style={/pgf/number format/1000 sep=},
	]
	\addplot[const plot,blue] file {wykresy/zadanie6_DMC_2reg_Y.txt};
	\addplot[red] file {wykresy/zadanie6_DMC_2reg_Yzad.txt};
	\legend{$Y$, $Y_{zad}$}
	\end{axis}
	\end{tikzpicture}
	
	\begin{tikzpicture}
	\begin{axis}[
	width=4.667in,
	height=1.8in,
	enlargelimits=false,
	%xmin=0,xmax=1600,
	ymin=0,ymax=100,
	xlabel={$k$},
	ylabel={$U$},
	legend pos=north east,
	y tick label style={/pgf/number format/1000 sep=},
	]
	\addplot[const plot,blue] file {wykresy/zadanie6_DMC_2reg_U.txt};
	\end{axis}
	\end{tikzpicture}
	
	\begin{tikzpicture}
	\begin{axis}[
	width=4.667in,
	height=1.8in,
	enlargelimits=false,
	%xmin=0,xmax=1600,
	ymin=-5,ymax=20,
	xlabel={$k$},
	ylabel={$E$},
	legend pos=north east,
	y tick label style={/pgf/number format/1000 sep=},
	]
	\addplot[const plot,blue] file {wykresy/zadanie6_DMC_2reg_blad.txt};
	\end{axis}
	\end{tikzpicture}
	
	\caption{Przebiegi sygna��w uzyskanych podczas eksperymentu na rzeczywistym obiekcie w przypadku zastosowania dw�ch regulator�w DMC}
\end{figure}
%	%LAB4
\chapter{Punkt 8}





	

\end{document}


