%LAB3
\chapter{Punkt 1 }

Obiekt po podaniu zerowego sterowania na wszystkie 4 wej�cia, nie zmienia swojego stanu, wszystkie wyj�cia s� w stanie ustalonym i maj�
warto�� $y_1$=$y_2$=$y_3$=0, co potwierdza prawdziwo�� punkt�w pracy $U_{pp}$ i $Y_{pp}$ dla ka�dego z wej��/wyj��.

\begin{figure}[ht]
	\centering
	\begin{tikzpicture}
	\begin{axis}[
	enlargelimits=false,
	width=5.0in,
	height=1.4in,
	%xmin=0,xmax=300,
	%ymin=32,ymax=34,
	%xlabel={$k$},
	ylabel={$Y1$},
	]
	\addplot[const plot,cyan] file {wykresy/zad1y1.txt};
	\end{axis}
	\end{tikzpicture}

	\begin{tikzpicture}
	\begin{axis}[
	enlargelimits=false,
	width=5.0in,
	height=1.4in,
	%xmin=0,xmax=300,
	%ymin=32,ymax=34,
	%xlabel={$k$},
	ylabel={$Y2$},
	]
	\addplot[const plot,cyan] file {wykresy/zad1y2.txt};
	\end{axis}
	\end{tikzpicture}
	
	\begin{tikzpicture}
	\begin{axis}[
	enlargelimits=false,
	width=5.0in,
	height=1.4in,
	%xmin=0,xmax=300,
	%ymin=32,ymax=34,
	xlabel={$k$},
	ylabel={$Y3$},
	]
	\addplot[const plot,cyan] file {wykresy/zad1y3.txt};
	\end{axis}
	\end{tikzpicture}

	\caption{Punkt pracy}
	
\end{figure}

\begin{figure}

	
\end{figure}