%PROJ3
\chapter{Punkt 7}
Szum addytywny zosta? wygenerowany jako jendnostowe skoki dodawane do pomiaru sygna?u wyj?ciowego. Mo?emy je obserwowa? na poni?szych wykresach. Mo?emy zauwa?y?, ?e dostrojone wcze?niej regulatory dobrze radz? sobie ze stabilizacj? zak?�ce?, nie wpadaj? w dragania. 
Poni?ej, jako przyk??d zosta? przedstawiony wykres dla regulatora PID w wersji 1.\\
e = $e_1$ + $e_2$ = \num{590,4457}


\begin{figure}[ht]
	\centering
	\begin{tikzpicture}
	\begin{axis}[
	width=7.0in,
	height=2.4in,
	xmin=0,xmax=1600,ymin=-2,ymax=2,
	xlabel={$k$},
	ylabel={$zaszumienie$},
	legend pos=south east,
	y tick label style={/pgf/number format/1000 sep=},
	]
	\addplot[const plot,cyan] file {wykresy/zadanie7_PID_V1_zaszumienie_Y1.txt};
	\end{axis}
	\end{tikzpicture}
	
	\begin{tikzpicture}
	\begin{axis}[
	width=7.0in,
	height=2.4in,
	xmin=0,xmax=1600,ymin=-1,ymax=4,
	xlabel={$k$},
	ylabel={$y_1$},
	legend pos=south east,
	y tick label style={/pgf/number format/1000 sep=},
	]
	\addplot[const plot,red, densely dashed] file {wykresy/zadanie7_PID_V1_Y1_zad.txt};
	\addplot[const plot,cyan] file {wykresy/zadanie7_PID_V1_Y1.txt};
	\legend{$y_1^{zad}$, $y_1$}
	\end{axis}
	\end{tikzpicture}
	
	\begin{tikzpicture}
	\begin{axis}[
	width=7.0in,
	height=2.4in,
	xmin=0,xmax=1600,ymin=-5,ymax=3,
	xlabel={$k$},
	ylabel={$e_1$},
	legend pos=south east,
	y tick label style={/pgf/number format/1000 sep=},
	]
	\addplot[const plot,cyan] file {wykresy/zadanie7_PID_V1_E1.txt};
	\end{axis}
	\end{tikzpicture}
	
	\caption{//TODO}
	\label{odpskokUdmc3}
\end{figure}


\begin{figure}[ht]
	\centering
	\begin{tikzpicture}
	\begin{axis}[
	width=7.0in,
	height=2.4in,
	xmin=0,xmax=1600,ymin=-2,ymax=2,
	xlabel={$k$},
	ylabel={$zaszumienie$},
	legend pos=south east,
	y tick label style={/pgf/number format/1000 sep=},
	]
	\addplot[const plot,cyan] file {wykresy/zadanie7_PID_V1_zaszumienie_Y2.txt};
	\end{axis}
	\end{tikzpicture}
	
	\begin{tikzpicture}
	\begin{axis}[
	width=7.0in,
	height=2.4in,
	xmin=0,xmax=1600,ymin=-1,ymax=4,
	xlabel={$k$},
	ylabel={$y_2$},
	legend pos=south east,
	y tick label style={/pgf/number format/1000 sep=},
	]
	\addplot[const plot,red, densely dashed] file {wykresy/zadanie7_PID_V1_Y2_zad.txt};
	\addplot[const plot,cyan] file {wykresy/zadanie7_PID_V1_Y2.txt};
	\legend{$y_2^{zad}$, $y_2$}
	\end{axis}
	\end{tikzpicture}
	
	\begin{tikzpicture}
	\begin{axis}[
	width=7.0in,
	height=2.4in,
	xmin=0,xmax=1600,ymin=-3,ymax=2,
	xlabel={$k$},
	ylabel={$e_2$},
	legend pos=south east,
	y tick label style={/pgf/number format/1000 sep=},
	]
	\addplot[const plot,cyan] file {wykresy/zadanie7_PID_V1_E2.txt};
	\end{axis}
	\end{tikzpicture}
	
	\caption{$y_2$}
	\label{odpskokUdmc3}
\end{figure}


\begin{figure}[ht]
	\centering
	\begin{tikzpicture}
	\begin{axis}[
	width=7.0in,
	height=2.4in,
	xmin=0,xmax=1600,ymin=-6,ymax=5,
	xlabel={$k$},
	ylabel={$u_1$},
	legend pos=south east,
	y tick label style={/pgf/number format/1000 sep=},
	]
	\addplot[const plot,cyan] file {wykresy/zadanie7_PID_V1_U1.txt};
	\end{axis}
	\end{tikzpicture}
	
	
	\begin{tikzpicture}
	\begin{axis}[
	width=7.0in,
	height=2.4in,
	xmin=0,xmax=1600,ymin=-2,ymax=5,
	xlabel={$k$},
	ylabel={$u_2$},
	legend pos=south east,
	y tick label style={/pgf/number format/1000 sep=},
	]
	\addplot[const plot,cyan] file {wykresy/zadanie7_PID_V1_U2.txt};
	\end{axis}
	\end{tikzpicture}
	
	\caption{///TODO}
	\label{odpskokUdmc3}
\end{figure}


\begin{figure}
	Oraz dla regulatora DMC:\\
	B��d e = $\num{1077,52}$
\end{figure}



\begin{figure}[ht]
	\centering
	\begin{tikzpicture}
	\begin{axis}[
	width=5.0in,
	height=1.4in,
	xmin=0,xmax=1600,ymin=0,ymax=1,
	xlabel={$k$},
	ylabel={$zaklocenie$},
	legend pos=south east,
	y tick label style={/pgf/number format/1000 sep=},
	]
	\addplot[const plot,cyan] file {wykresy/zadanie7_dmc_zaklocenie.txt};
	\end{axis}
	\end{tikzpicture}
	
	\begin{tikzpicture}
	\begin{axis}[
	width=5.0in,
	height=1.4in,
	xmin=0,xmax=1600,ymin=-0.5,ymax=15,
	xlabel={$k$},
	ylabel={$y_1$},
	legend pos=south east,
	y tick label style={/pgf/number format/1000 sep=},
	]
	\addplot[const plot,cyan] file {wykresy/zadanie7_dmc_Y1_addytywne.txt};
	\end{axis}
	\end{tikzpicture}
	
	\begin{tikzpicture}
	\begin{axis}[
	width=5.0in,
	height=1.4in,
	xmin=0,xmax=1600,ymin=-0.5,ymax=22,
	xlabel={$k$},
	ylabel={$y_2$},
	legend pos=south east,
	y tick label style={/pgf/number format/1000 sep=},
	]
	\addplot[const plot,cyan] file {wykresy/zadanie7_dmc_Y2_addytywne.txt};
	\end{axis}
	\end{tikzpicture}
	
	\begin{tikzpicture}
	\begin{axis}[
	width=5.0in,
	height=1.4in,
	xmin=0,xmax=1600,ymin=-40,ymax=2,
	xlabel={$k$},
	ylabel={$u_1$},
	legend pos=south east,
	y tick label style={/pgf/number format/1000 sep=},
	]
	\addplot[const plot,cyan] file {wykresy/zadanie7_dmc_U1_addytywne.txt};
	\end{axis}
	\end{tikzpicture}
	
	\begin{tikzpicture}
	\begin{axis}[
	width=5.0in,
	height=1.4in,
	xmin=0,xmax=1600,ymin=-40,ymax=1,
	xlabel={$k$},
	ylabel={$u_2$},
	legend pos=south east,
	y tick label style={/pgf/number format/1000 sep=},
	]
	\addplot[const plot,cyan] file {wykresy/zadanie7_dmc_U2_addytywne.txt};
	\end{axis}
	\end{tikzpicture}
	
	\caption{Addytywne}
	\label{odpskokUdsrfsrfmc3}
\end{figure}

