\chapter{Punkt 3}



Odpowiedzi skokowe wyznaczamy analogicznie do poprzedniego projektu, jedyn� zmian� jest to, �e w przypadku obliczania odpowiedzi skokowej dla skoku sygna�u steruj�cego, zerujemy warto�� zak��ce�, a w przypadku obliczania odpowiedzi skokowej od zak��ce? zerujemy sygna� steruj�cy.

\begin{figure}[ht]
	\centering
	\begin{tikzpicture}
	\begin{axis}[
	width=5.0in,
	height=2.4in,
	xmin=0,xmax=130,ymin=0,ymax=2.5,
	xlabel={$k$},
	ylabel={$Y$},
	legend pos=south east,
	y tick label style={/pgf/number format/1000 sep=},
	]
	\addplot[const plot,cyan] file {wykresy/zadanie3_odpSkok_U_schodki.txt};
	\end{axis}
	\end{tikzpicture}
	\begin{tikzpicture}
	\begin{axis}[
	width=5.0in,
	height=2.4in,
	xmin=0,xmax=130,ymin=0,ymax=1.5,
	xlabel={$k$},
	ylabel={$Y$},
	legend pos=south east,
	y tick label style={/pgf/number format/1000 sep=},
	]
	\addplot[const plot,cyan] file {wykresy/zadanie3_odpSkok_Z_schodki.txt};
	\end{axis}
	\end{tikzpicture}
	\caption{Odpowied� skokowa toru wej�cie-wyj�cie (g�ra) oraz  zak��cenie-wyj�cie (d�)}
	\label{odpskokUdmc}
\end{figure}
