\documentclass[a4paper,titlepage,11pt,twosides,floatssmall]{mwrep}
\usepackage[left=2.5cm,right=2.5cm,top=2.5cm,bottom=2.5cm]{geometry}
\usepackage[OT1]{fontenc}
\usepackage{polski}
\usepackage{amsmath}
\usepackage{amsfonts}
\usepackage{amssymb}
\usepackage{graphicx}
\usepackage{url}
\usepackage{tikz}
\usetikzlibrary{arrows,calc,decorations.markings,math,arrows.meta}
\usepackage{rotating}
\usepackage[percent]{overpic}
\usepackage[cp1250]{inputenc}
\usepackage{xcolor}
\usepackage{pgfplots}
\usetikzlibrary{pgfplots.groupplots}
\usepackage{listings}
\usepackage{matlab-prettifier}
\usepackage{siunitx}
\usepackage{verbatim}
\definecolor{szary}{rgb}{0.95,0.95,0.95}
\usepackage{gensymb}
%dodane z pliku zapisz_pdf_symulacje22.tex
\begin{comment}
\pgfplotsset{compat=1.11}
\usepgfplotslibrary{external}
%\tikzexternalize
\textwidth 160mm \textheight 247mm


\newcommand{\szer}{16cm}
\newcommand{\wys}{6cm}
\newcommand{\odstepionowy}{1.2cm}
\end{comment}


\sisetup{detect-weight,exponent-product=\cdot,output-decimal-marker={,},per-mode=symbol,binary-units=true,range-phrase={-},range-units=single}
\SendSettingsToPgf
%konfiguracje pakietu listings
\lstset{
	backgroundcolor=\color{szary},
	frame=single,
	breaklines=true,
}
\lstdefinestyle{customlatex}{
	basicstyle=\footnotesize\ttfamily,
	%basicstyle=\small\ttfamily,
}
\lstdefinestyle{customc}{
	breaklines=true,
	frame=tb,
	language=C,
	xleftmargin=0pt,
	showstringspaces=false,
	basicstyle=\small\ttfamily,
	keywordstyle=\bfseries\color{green!40!black},
	commentstyle=\itshape\color{purple!40!black},
	identifierstyle=\color{blue},
	stringstyle=\color{orange},
}
\lstdefinestyle{custommatlab}{
	captionpos=t,
	breaklines=true,
	frame=tb,
	xleftmargin=0pt,
	language=matlab,
	showstringspaces=false,
	%basicstyle=\footnotesize\ttfamily,
	basicstyle=\scriptsize\ttfamily,
	keywordstyle=\bfseries\color{green!40!black},
	commentstyle=\itshape\color{purple!40!black},
	identifierstyle=\color{blue},
	stringstyle=\color{orange},
}

%wymiar tekstu (bez �ywej paginy)
\textwidth 160mm \textheight 247mm

%ustawienia pakietu pgfplots
\pgfplotsset{
tick label style={font=\scriptsize},
label style={font=\small},
legend style={font=\small},
title style={font=\small}
}

\def\figurename{Rys.}
\def\tablename{Tab.}

%konfiguracja liczby p�ywaj�cych element�w
\setcounter{topnumber}{0}%2
\setcounter{bottomnumber}{3}%1
\setcounter{totalnumber}{5}%3
\renewcommand{\textfraction}{0.01}%0.2
\renewcommand{\topfraction}{0.95}%0.7
\renewcommand{\bottomfraction}{0.95}%0.3
\renewcommand{\floatpagefraction}{0.35}%0.5

\begin{document}
\frenchspacing
\pagestyle{uheadings}

%strona tytu�owa
\title{\bf Sprawozdanie z �wiczenia laboratoryjnego nr 1, zadanie nr 3\vskip 0.1cm}
\author{Bart�omiej Boczek, Aleksander Piotrowski, �ukasz �migielski}
\date{2017}

\makeatletter
\renewcommand{\maketitle}{\begin{titlepage}
\begin{center}{\LARGE {\bf
Wydzia� Elektroniki i Technik Informacyjnych}}\\
\vspace{0.4cm}
{\LARGE {\bf Politechnika Warszawska}}\\
\vspace{0.3cm}
\end{center}
\vspace{5cm}
\begin{center}
{\bf \LARGE Projektowanie uk�ad�w sterowania\\ (projekt grupowy) \vskip 0.1cm}
\end{center}
\vspace{1cm}
\begin{center}
{\bf \LARGE \@title}
\end{center}
\vspace{2cm}
\begin{center}
{\bf \Large \@author \par}
\end{center}
\vspace*{\stretch{6}}
\begin{center}
\bf{\large{Warszawa, \@date\vskip 0.1cm}}
\end{center}
\end{titlepage}
}
\makeatother

\maketitle

\tableofcontents
\chapter{Punkt 1}
%Wyjscie procesu zbiega do $Y_{\mathrm{pp}}$ przy sygnale sterujacym rownym  $U_{\mathrm{pp}}$. Potwierdza to poprawnosc wartosci punktu pracy.

Na samym pocz�tku �wiczenia laboratoryjnego sprawdzili�my, �e jest mo�liwo�� sterowania obiektem oraz pomiaru sygna�u wyj�ciowego. 
Dla mocy grza�ki $G1=\num{28}\%$ warto�� temperatury w punkcie pracy ustali�a si� na poziomie $\num{30,87}$ $\degree$C. Warto�� mocy wentylatora $W1$ przez ca�y przebieg laboratorium wynosi�a $\num{50}\%$.

\chapter{Punkt 2}
Z punktu pracy $G1=\num{28}\%$ wybrali�my trzy skoki sygna�u steruj�cego: do $\num{50}\%$, $\num{70}\%$ oraz $\num{90}\%$. 

\begin{comment}
\begin{figure}[h]
	\centering
	\begin{tikzpicture}
	\begin{axis}[
	width=0.65\textwidth,
	xmin=1,xmax=350,ymin=30,ymax=50,
	xlabel={$sek$},
	ylabel={$y$},
	xtick={1,100, 200, 300, 350},
	ytick={30, 35, 40, 45, 50},
	legend pos=south east,
	y tick label style={/pgf/number format/1000 sep=},
	]
	\addplot[const plot, red,semithick] file {wykresy/lab1/Y2_50.txt};
	\addplot[const plot, blue,semithick] file {wykresy/lab1/Y2_60.txt};
	\addplot[const plot, color=green,semithick] file {wykresy/lab1/Y2_90.txt};
	\legend{$G1=\num{50}\%$, $G1=\num{70}\%$, $G1=\num{90}\%$}
	\end{axis}
	\end{tikzpicture}
	\caption{Odpowiedzi skokowe dla trzech zmian sygna�u steruj�cego}
	\label{r_pgfplots_funkcje}
\end{figure}
\end{comment}



Nie zrobili�my odpowiedniego eksperymentu (nale�a�o liniowo zwi�ksza� sterowanie), 
dlatego te� nie mo�emy stwierdzi�, na podstawie naszych danych wyniesionych ze stanowiska laboratoryjnego, czy w�a�ciwo�ci statyczne obiektu s� liniowe i jakie jest wzmocnienie statyczne.


\chapter{Punkt 3}
 
W tym zadaniu szukali�my optymalnej aproksymacji otrzymanej przez nas odpowiedzi skokowej dla skoku z warto�ci $U_{\mathrm{pp}}=\num{28}$ do $U = \num{50}$.
Podczas wyznaczania funkcji aproksymuj�cej optymalizowali�my parametry $T_{\mathrm{1}}$, $T_{\mathrm{2}}$ oraz $K$ cz�onu inercyjnego drugiego rz�du z op�nieniem. Optymalizacja polega�a na minimalizacji b��du �redniokwadratowego mi�dzy rzeczywist� odpowiedzi� skokow� a funkcj� aproksymuj�c�.
Przy u�yciu optymalizatora \verb+ga+ uda�o nam si� uzyska� funkcj�, dla kt�rej b��d wyni�s� $E=\num{1,8765}$, przy warto�ciach parametr�w: $T_{\mathrm{1}}=\num{17,099506}$; $T_{\mathrm{2}}=\num{64,970974}$; $K=\num{0,263644}$. Warto�� $T_{\mathrm{d}}$ (op�nienia) eksperymentalnie ustawili�my na \num{10}, gdy� dla niej wyszed� najmniejszy b��d aproksymacji. Poni�ej przedstawione zosta�y wykresy odpowiedzi skokowej oraz funckji aproksymuj�cej.

\begin{comment}
\begin{figure}[h]
	\centering
	\begin{tikzpicture}
	\begin{axis}[
	width=0.5\textwidth,
	xmin=1,xmax=130,ymin=0,ymax=3.5,
	xlabel={$k$},
	ylabel={$y$},
	xtick={1, 50, 100, 130},
	ytick={0, 1, 2, 3, 3.5},
	legend pos=south east,
	y tick label style={/pgf/number format/1000 sep=},
	]
	\addplot[const plot,color=blue,semithick] file {wykresy/zadanie3_odpSkok_schodki.txt};	
	\end{axis}
	\end{tikzpicture}
	\caption{Odpowiedz SKOKOWA}
	\label{r_pgfplots_funkcje}
\end{figure}	
\end{comment}
\begin{comment}
\begin{figure}[h]
	\centering
	\begin{tikzpicture}
	\begin{axis}[
	width=0.7\textwidth,
	xmin=1,xmax=350,ymin=-0.05,ymax=0.3,
	xlabel={$sek$},
	ylabel={$y$},
	xtick={1,100, 200, 300, 350},
	ytick={-0.05, 0, 0.05, 0.1, 0.15, 0.2, 0.25, 0.3},
	legend pos=south east,
	y tick label style={/pgf/number format/1000 sep=},
	]
	\addplot[const plot,color=blue,semithick] file {wykresy/lab1/odp_skokowa.txt};	
	\addplot[color=red, semithick, densely dashed] file {wykresy/lab1/funkcja_aproks.txt};
	\legend{orygina�, aproksymacja}
	\end{axis}
	\end{tikzpicture}
	\caption{Odpowiedz skokowa oryginalna i wersja aproksymowana}
	\label{r_pgfplots_funkcje}
\end{figure}
\end{comment}




\chapter{Punkt 4 i 5}

Z tym punktem nie poradzili�my sobie w regulaminowych 3 godzinach zaj��. Mamy jednak �wiadomo��, jak na podstawie odpowiedzi skokowej mo�na obliczy� parametry algorytmu DMC, a te z algorytmu PID okre�li� metod� eksperymentaln� lub in�yniersk�, wykorzystuj�c skrypty z oddanego wcze�niej projektu 1. Wystarczy�oby pomin�� ograniczenia szybko�ci zmian warto�ci steruj�cej, a ograniczenia warto�ci zadanej ustali� na $\num{0}< U < \num{100}$. 

\begin{comment}

Skrypt symulujacy cyfrowe algorytmy PID i DMC jest zaimplementowany w plikach zadanie45PID.m i DMC.m. 
Postac dyskretna algorytmu PID: 
\begin{equation}
r_2 = \frac{K T_{\mathrm{d}}}{T_{\mathrm{s}}} \
r_1 = K \left( \frac{T_{\mathrm{s}}}{2 T_{\mathrm{i}}} - \frac{2 T_{\mathrm{d}}}{T_{\mathrm{s}}} - 1 \right) \
r_0 = K \left( 1 + \frac{T_{\mathrm{s}}}{2 T_{\mathrm{i}}} + \frac{T_{\mathrm{d}}}{T_{\mathrm{s}}} \right) \\
\label{r_t}
\end{equation}

gdzie K to wzmocnienie regulatora, $T_{\mathrm{i}}$ stala calkowania, $T_{\mathrm{d}}$ stala rozniczkowania, $T_{\mathrm{s}}$ okres probkowania. Dodatkowo zgodnie z poleceniem uwzglednilismy ograniczenia zmiany wartosci sterowania $\num{-0,05}$  $\le$ $\triangle U(k)$ $\le$ $\num{0,05}$.

Nastawy regulatorow zostaly dobrane metoda eksperymentalna. W regulatorze PID wynosza one: K =  $\num{0,6}$; $T_{\mathrm{i}}$ = 11; $T_{\mathrm{d}}$ =  $\num{3}$; $T_{\mathrm{s}}$ = $\num{0,5}$;. Dla symulacji trwajacej 2500 probek, wartosc bledu sredniokwadratowego, czyli wskaznika jakosci regulacji, wynosi E= $\num{81,1977}$. Przerywana linia oznaczylismy wartosc zadana regulatora, niebieskimi, wartosci wyjscia i sterowania. 



\begin{figure}[h]
	\centering
	\begin{tikzpicture}
%	\begin{groupplot}[group style={group size=1 by 1,vertical sep=\odstepionowy},width=\szer,height=\wys]
	%%1
%	\nextgroupplot
	\begin{axis}[
	width=0.8\textwidth,
	height=0.19\textheight,
	xmin=1,xmax=2500,ymin=1,ymax=3,
	%xlabel={$k$},
	ylabel={$y$},
	xtick={1, 1000, 2000, 2500},
	ytick={1,2,3},
	legend pos=north west,
	y tick label style={/pgf/number format/1000 sep=},
	]
	\addplot[blue,semithick] file {wykresy/zadanie45_PID_Y.txt};
	\addplot[red , semithick , densely dashed] file {wykresy/zadanie45_PID_Yzad.txt};

	\legend{$Y$, $Y_{\mathrm{zad}}$}
	\end{axis}
	%\end{groupplot}
	\end{tikzpicture}
	%\caption{PID???}
	\label{r_pgfplots_funkcje}
\end{figure}



\begin{figure}[h]
	\centering
	\begin{tikzpicture}
	%	\begin{groupplot}[group style={group size=1 by 1,vertical sep=\odstepionowy},width=\szer,height=\wys]
	%%1
	%	\nextgroupplot
	\begin{axis}[
	width=0.8\textwidth,
	height=0.17\textheight,
	xmin=1,xmax=2500,ymin=0.7,ymax=1.5,
	xlabel={$k$},
	ylabel={$u$},
	xtick={1, 1000, 2000, 2500},
	ytick={0.7, 1, 1.5},
	legend pos=south east,
	y tick label style={/pgf/number format/1000 sep=},
	]
	\addplot[blue,semithick] file {wykresy/zadanie45_PID_U.txt};

	\end{axis}
	%\end{groupplot}
	\end{tikzpicture}
	\caption{PID}
	\label{r_pgfplots_funkcje}
\end{figure}



W regulatorze DMC obliczone nastawy wynosza: D=110; N=19; Nu=6; $\lambda$=$\num{0,15}$. Dla symulacji o takiej samej dlugosci jak w przypadku PID (2500 probek) wartosc bledu wynosi E=$\num{69,3695}$. Zgodnie z oczekiwaniami blad ten jest znaczaco mniejszy niz ten otrzymany przy regulacji PIDem. 

%%%%%%%%%%%To samo co wyzej ale na DMC zmien::
\begin{figure}[h]
	\centering
	\begin{tikzpicture}
	%	\begin{groupplot}[group style={group size=1 by 1,vertical sep=\odstepionowy},width=\szer,height=\wys]
	%%1
	%	\nextgroupplot
	\begin{axis}[
	width=0.8\textwidth,
	height=0.19\textheight,
	xmin=1,xmax=2500,ymin=1,ymax=3,
	%xlabel={$k$},
	ylabel={$y$},
	xtick={1, 1000, 2000, 2500},
	ytick={1,2,3},
	legend pos=north west,
	y tick label style={/pgf/number format/1000 sep=},
	]
	\addplot[blue,semithick] file {wykresy/zadanie45_DMC_Y.txt};
	\addplot[red , semithick , densely dashed] file {wykresy/zadanie45_DMC_Yzad.txt};
	
	\legend{$Y$, $Y_{\mathrm{zad}}$}
	\end{axis}
	%\end{groupplot}
	\end{tikzpicture}
	%\caption{PID???}
	\label{r_pgfplots_funkcje}
\end{figure}



\begin{figure}[h]
	\centering
	\begin{tikzpicture}
	%	\begin{groupplot}[group style={group size=1 by 1,vertical sep=\odstepionowy},width=\szer,height=\wys]
	%%1
	%	\nextgroupplot
	\begin{axis}[
	width=0.8\textwidth,
	height=0.17\textheight,
	xmin=1,xmax=2500,ymin=0.7,ymax=1.5,
	xlabel={$k$},
	ylabel={$u$},
	xtick={1, 1000, 2000, 2500},
	ytick={0.7, 1, 1.5},
	legend pos=south east,
	y tick label style={/pgf/number format/1000 sep=},
	]
	\addplot[blue,semithick] file {wykresy/zadanie45_DMC_U.txt};
	
	\end{axis}
	%\end{groupplot}
	\end{tikzpicture}
	\caption{DMC}
	\label{r_pgfplots_funkcje}
\end{figure}







\chapter{Punkt 5}

Om�wi� wyniki i ewentualne sposoby poprawy jako�ci regulacji. Jako��
regulacji ocenia� jako�ciowo (na podstawie rysunk�w przebieg�w sygna��w) oraz ilo-
�ciowo, wyznaczaj�c wska�nik jako�ci regulacji.


\begin{figure}[h]
	\centering
	\begin{tikzpicture}
	%	\begin{groupplot}[group style={group size=1 by 1,vertical sep=\odstepionowy},width=\szer,height=\wys]
	%%1
	%	\nextgroupplot
	\begin{axis}[
	width=0.8\textwidth,
	height=0.19\textheight,
	xmin=1,xmax=2500,ymin=1,ymax=3,
	%xlabel={$k$},
	ylabel={$y$},
	xtick={1, 1000, 2000, 2500},
	ytick={1,2,3},
	legend pos=north west,
	y tick label style={/pgf/number format/1000 sep=},
	]
	\addplot[blue,semithick] file {wykresy/zadanie6_PID_Y.txt};
	\addplot[red , semithick , densely dashed] file {wykresy/zadanie6_PID_Yzad.txt};
	
	\legend{$Y$, $Y_{\mathrm{zad}}$}
	\end{axis}
	%\end{groupplot}
	\end{tikzpicture}
	%\caption{PID???}
	\label{r_pgfplots_funkcje}
\end{figure}



\begin{figure}[h]
	\centering
	\begin{tikzpicture}
	%	\begin{groupplot}[group style={group size=1 by 1,vertical sep=\odstepionowy},width=\szer,height=\wys]
	%%1
	%	\nextgroupplot
	\begin{axis}[
	width=0.8\textwidth,
	height=0.17\textheight,
	xmin=1,xmax=2500,ymin=0.7,ymax=1.5,
	xlabel={$k$},
	ylabel={$u$},
	xtick={1, 1000, 2000, 2500},
	ytick={0.7, 1, 1.5},
	legend pos=south east,
	y tick label style={/pgf/number format/1000 sep=},
	]
	\addplot[blue,semithick] file {wykresy/zadanie6_PID_U.txt};
	
	\end{axis}
	%\end{groupplot}
	\end{tikzpicture}
	\caption{PID}
	\label{r_pgfplots_funkcje}
\end{figure}





%%%%%%%%%%%To samo co wyzej ale na DMC zmien::
\begin{figure}[h]
	\centering
	\begin{tikzpicture}
	%	\begin{groupplot}[group style={group size=1 by 1,vertical sep=\odstepionowy},width=\szer,height=\wys]
	%%1
	%	\nextgroupplot
	\begin{axis}[
	width=0.8\textwidth,
	height=0.19\textheight,
	xmin=1,xmax=2500,ymin=1,ymax=3,
	%xlabel={$k$},
	ylabel={$y$},
	xtick={1, 1000, 2000, 2500},
	ytick={1,2,3},
	legend pos=north west,
	y tick label style={/pgf/number format/1000 sep=},
	]
	\addplot[blue,semithick] file {wykresy/zadanie6_DMC_Y.txt};
	\addplot[red , semithick , densely dashed] file {wykresy/zadanie6_DMC_Yzad.txt};
	
	\legend{$Y$, $Y_{\mathrm{zad}}$}
	\end{axis}
	%\end{groupplot}
	\end{tikzpicture}
	%\caption{PID???}
	\label{r_pgfplots_funkcje}
\end{figure}



\begin{figure}[h]
	\centering
	\begin{tikzpicture}
	%	\begin{groupplot}[group style={group size=1 by 1,vertical sep=\odstepionowy},width=\szer,height=\wys]
	%%1
	%	\nextgroupplot
	\begin{axis}[
	width=0.8\textwidth,
	height=0.17\textheight,
	xmin=1,xmax=2500,ymin=0.7,ymax=1.5,
	xlabel={$k$},
	ylabel={$u$},
	xtick={1, 1000, 2000, 2500},
	ytick={0.7, 1, 1.5},
	legend pos=south east,
	y tick label style={/pgf/number format/1000 sep=},
	]
	\addplot[blue,semithick] file {wykresy/zadanie6_DMC_U.txt};
	
	\end{axis}
	%\end{groupplot}
	\end{tikzpicture}
	\caption{DMC}
	\label{r_pgfplots_funkcje}
\end{figure}

Podczas optymalizacji regulatora DMC optymalizowalismy tylko parametry $N_{\mathrm{u}}$, N i $\lambda$. Parametr D zostal ustawiony na stala wartosc 110. Dla nastaw regulatora DMC otrzymanch w wyniku optymaliacji D =110; N=130; $N_{\mathrm{u}}$=6; $\lambda$=$\num{0,01}$ otrzymalismy blad sredniokwadratowy r�wny E =$\num{69,3671}$ , co jest bardzo nieznacznie mniejsza wartoscia niz ta obliczona eksperymantalnie, a przebiegi zdradzaja, ze w tym przypadku optymalizacja okazala sie lepsza niz dla PIDa, uklad nie wpada w oscylacje a regulacja wydaje sie byc ogolnie nieco lepsza niz w punkcie 5.
Na podstawie powyzszych eksperymentow mozemy stwierdzic, ze nie zawsze metody matematyczne daja lepsze rezultaty. Przypadek PIDa pokazal, ze pomimo mniejszej wartosci bledu regulacja byla gorsza z powodu ograniczen, dlatego do wynikow metod matematycznych nalezy podchodzic z pewnym pragmatyzmem.
\end{comment}

\end{document}


