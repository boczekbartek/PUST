\chapter{Punkt 1 }


Z poni�szych wykres�w mo�emy odczyta�, �e dla punktu pracy $U=Z=0$ warto�� wyj�cia $Y$ jest sta�a i ma warto�� $Y=0$, co dowodzi poprawno�ci wybrania punktu pracy.


\begin{figure}[ht]
	\centering
	\begin{tikzpicture}
	\begin{axis}[
	width=4.667in,
	height=1.645in,
	xmin=0,xmax=300,ymin=-1,ymax=1,
	xlabel={$k$},
	ylabel={$Y$},
	legend pos=south east,
	y tick label style={/pgf/number format/1000 sep=},
	]
	\addplot[const plot,blue] file {wykresy/zadanie1Y.txt};
	\end{axis}
	\end{tikzpicture}
	\begin{tikzpicture}
	\begin{axis}[
	width=4.667in,
	height=1.645in,
	xmin=0,xmax=300,ymin=-1,ymax=1,
	xlabel={$k$},
	ylabel={$U$},
	legend pos=south east,
	y tick label style={/pgf/number format/1000 sep=},
	]
	\addplot[const plot,blue] file {wykresy/zadanie1U.txt};
	\end{axis}
	\end{tikzpicture}
	\begin{tikzpicture}
	\begin{axis}[
	width=4.667in,
	height=1.645in,
	xmin=0,xmax=300,ymin=-1,ymax=1,
	xlabel={$k$},
	ylabel={$Z$},
	legend pos=south east,
	y tick label style={/pgf/number format/1000 sep=},
	]
	\addplot[const plot,red] file {wykresy/zadanie1Z.txt};
	\end{axis}
	\end{tikzpicture}
	\caption{Sygna�y w punkcie pracy}
	\label{r_pgfplots_funkcje}
\end{figure}
