\chapter{Punkt 6}

Dla zak��cenia sinusoidalnego $Z=sin(0:0.01:25)$; w��czonego do obiektu w momencie $k=80$; regulacja z pomiarem zak��cenia jest znacznie lepsza, mimo �e regulator DMC nie jest w stanie ca�kowicie wyeliminowa� oscylacji.
Bez pomiaru b��d $E = \num{49,1146}$; z pomiarem $E = \num{33,0742}$. 



Na poni�szych wykresach wida�, �e regulacja z pomiarem zak��cenia powoduje wyg�adzenie oscylacji kt�re wprowadza zak��cenie.

\begin{figure}[h!]
	\centering
	
	\begin{tikzpicture}
	\begin{axis}[
	width=4.667in,
	height=1.2in,
	xmin=0,xmax=2500,ymin=-0.2,ymax=4,
	%ytick={0,0,2},
	xlabel={$k$},
	ylabel={$Y$},
	legend pos=south east,
	y tick label style={/pgf/number format/1000 sep=},
	]
	\addplot[const plot,blue] file {wykresy/zadanie6_bezpomiaru_DMC_Y.txt};
	\addplot[const plot,red,densely dashed] file {wykresy/zadanie6_bezpomiaru_DMC_Yzad.txt};
	\legend{$Y(k)$,$Y^{zad}(k)$}
	\end{axis}
	\end{tikzpicture}
	
	\begin{tikzpicture}
	\begin{axis}[
	width=4.667in,
	height=1.2in,
	xmin=0,xmax=2500,ymin=-1,ymax=3.5,
	%ytick={-3,0,3},
	xlabel={$k$},
	ylabel={$U$},
	legend pos=south east,
	y tick label style={/pgf/number format/1000 sep=},
	]
	\addplot[const plot,blue] file {wykresy/zadanie6_bezpomiaru_DMC_U.txt};
	\end{axis}
	\end{tikzpicture}
	\begin{tikzpicture}
	\begin{axis}[
	width=4.667in,
	height=1.1in,
	xmin=0,xmax=2500,ymin=-1.5,ymax=1.5,
	ytick={0,0,1.5},
	xlabel={$k$},
	ylabel={$Z$},
	legend pos=south east,
	y tick label style={/pgf/number format/1000 sep=},
	]
	\addplot[const plot,blue] file {wykresy/zadanie6_bezpomiaru_DMC_Z.txt};
	\end{axis}
	\end{tikzpicture}
	
	
	\caption{Odpowied� skokowa toru wej�cie-wyj�cie w wersji bez pomiaru}
	\label{bezPomZak}
\end{figure}
\FloatBarrier


\begin{figure}[h!]
	\centering
	
	\begin{tikzpicture}
	\begin{axis}[
	width=4.667in,
	height=1.2in,
	xmin=0,xmax=2500,ymin=-0.2,ymax=4,
	%ytick={0,0,2},
	xlabel={$k$},
	ylabel={$Y$},
	legend pos=south east,
	y tick label style={/pgf/number format/1000 sep=},
	]
	\addplot[const plot,blue] file {wykresy/zadanie6_pomiar_DMC_Y.txt};
	\addplot[const plot,red, densely dashed] file {wykresy/zadanie6_pomiar_DMC_Yzad.txt};
	\legend{$Y(k)$,$Y^{zad}(k)$}
	\end{axis}
	\end{tikzpicture}
	
	\begin{tikzpicture}
	\begin{axis}[
	width=4.667in,
	height=1.2in,
	xmin=0,xmax=2500,ymin=-2,ymax=3.5,
	%ytick={-3,0,3},
	xlabel={$k$},
	ylabel={$U$},
	legend pos=south east,
	y tick label style={/pgf/number format/1000 sep=},
	]
	\addplot[const plot,blue] file {wykresy/zadanie6_pomiar_DMC_U.txt};
	\end{axis}
	\end{tikzpicture}
	\begin{tikzpicture}
	\begin{axis}[
	width=4.667in,
	height=1.1in,
	xmin=0,xmax=2500,ymin=-1.5,ymax=1.5,
	%	ytick={0,0,1.5},
	xlabel={$k$},
	ylabel={$Z$},
	legend pos=south east,
	y tick label style={/pgf/number format/1000 sep=},
	]
	\addplot[const plot,blue] file {wykresy/zadanie6_pomiar_DMC_Z.txt};
	\end{axis}
	\end{tikzpicture}
	
	
	\caption{Odpowied� skokowa toru wej�cie-wyj�cie w wersji z pomiarem}
	\label{bezPomZak}
\end{figure}
\FloatBarrier

