\chapter{Punkt 5}

Skoki zak��cenia z warto�ci 0 do 1 nast�puj� w chwilach:\\
k=160 powr�t z warto�ci 1 do 0 w k=800\\
k=1040 powr�t z warto�ci 1 do 0 w k=1200\\
k=1640 powr�t z warto�ci 1 do 0 w k=2000\\
k=2290 \\

Warto�� parametru $D^{\mathrm{z}}$ = $\num{50}$ zosta�a dobrana eksperymentalnie. Z rysunk�w wynika, �e regulacja z pomiarem zak��cenia polepsza si� jako�ciowo (skoki wyj�cia s� mniejsze) i ilo�ciowo.
B��d $E$ maleje z $\num{71,1137}$ w przypadku bez pomiaru do $\num{41,6930}$ w przypadku z pomiarem.

\begin{figure}[h!]
	\centering
	
	\begin{tikzpicture}
	\begin{axis}[
	width=4.667in,
	height=1.2in,
	xmin=0,xmax=2500,ymin=-0.2,ymax=4,
	%ytick={0,0,2},
	xlabel={$k$},
	ylabel={$Y$},
	legend pos=south east,
	y tick label style={/pgf/number format/1000 sep=},
	]
	\addplot[const plot,blue] file {wykresy/zadanie5_bezpomiaru_DMC_Y.txt};
	\addplot[const plot,red,densely dashed] file {wykresy/zadanie5_bezpomiaru_DMC_Yzad.txt};
	\legend{$Y(k)$,$Y^{zad}(k)$}
	\end{axis}
	\end{tikzpicture}
	
	\begin{tikzpicture}
	\begin{axis}[
	width=4.667in,
	height=1.2in,
	xmin=0,xmax=2500,ymin=-1,ymax=3.5,
	%ytick={-3,0,3},
	xlabel={$k$},
	ylabel={$U$},
	legend pos=south east,
	y tick label style={/pgf/number format/1000 sep=},
	]
	\addplot[const plot,blue] file {wykresy/zadanie5_bezpomiaru_DMC_U.txt};
	\end{axis}
	\end{tikzpicture}
	\begin{tikzpicture}
	\begin{axis}[
	width=4.667in,
	height=1.1in,
	xmin=0,xmax=2500,ymin=-0.5,ymax=1.5,
	ytick={0,0,1.5},
	xlabel={$k$},
	ylabel={$Z$},
	legend pos=south east,
	y tick label style={/pgf/number format/1000 sep=},
	]
	\addplot[const plot,blue] file {wykresy/zadanie5_bezpomiaru_DMC_Z.txt};
	\end{axis}
	\end{tikzpicture}


	\caption{Odpowied� skokowa toru wej�cie-wyj�cie w wersji bez pomiaru}
	\label{bezPomZak}
\end{figure}
\FloatBarrier


\begin{figure}[h!]
	\centering
	
	\begin{tikzpicture}
	\begin{axis}[
	width=4.667in,
	height=1.2in,
	xmin=0,xmax=2500,ymin=-0.2,ymax=4,
	%ytick={0,0,2},
	xlabel={$k$},
	ylabel={$Y$},
	legend pos=south east,
	y tick label style={/pgf/number format/1000 sep=},
	]
	\addplot[const plot,blue] file {wykresy/zadanie5_pomiar_DMC_Y.txt};
	\addplot[const plot,red, densely dashed] file {wykresy/zadanie5_pomiar_DMC_Yzad.txt};
	\legend{$Y(k)$,$Y^{zad}(k)$}
	\end{axis}
	\end{tikzpicture}
	
	\begin{tikzpicture}
	\begin{axis}[
	width=4.667in,
	height=1.2in,
	xmin=0,xmax=2500,ymin=-2,ymax=3.5,
	%ytick={-3,0,3},
	xlabel={$k$},
	ylabel={$U$},
	legend pos=south east,
	y tick label style={/pgf/number format/1000 sep=},
	]
	\addplot[const plot,blue] file {wykresy/zadanie5_pomiar_DMC_U.txt};
	\end{axis}
	\end{tikzpicture}
	\begin{tikzpicture}
	\begin{axis}[
	width=4.667in,
	height=1.1in,
	xmin=0,xmax=2500,ymin=-0.5,ymax=1.5,
%	ytick={0,0,1.5},
	xlabel={$k$},
	ylabel={$Z$},
	legend pos=south east,
	y tick label style={/pgf/number format/1000 sep=},
	]
	\addplot[const plot,blue] file {wykresy/zadanie5_pomiar_DMC_Z.txt};
	\end{axis}
	\end{tikzpicture}
	
	
	\caption{Odpowied� skokowa toru wej�cie-wyj�cie w wersji z pomiarem}
	\label{bezPomZak}
\end{figure}
\FloatBarrier


