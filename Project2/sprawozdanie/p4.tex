\chapter{Punkt 4}


Parametry regulatora DMC: $D=\num{110}$; $N=\num{130}$; $N_u=\num{130}$; $\lambda=\num{0,92}$ dobrane metod� eksperymentaln� oceniaj�c jako�� regulacji jako�ciowo, aby nie by�o oscylacji przy skokach sterowania oraz minimalizuj�c b��d �redniokwadratowy. Dla d�ugo�ci symulacji r�wnej $n=\num{2500}$
warto�� b��du $E=\num{39,8848}$.

\begin{figure}[h!]
	\centering
	\begin{tikzpicture}
	\begin{axis}[
	width=4.667in,
	height=1.645in,
	xmin=0,xmax=2500,ymin=-0.1,ymax=3,
	xlabel={$k$},
	ylabel={$Y$},
	legend pos=south east,
	y tick label style={/pgf/number format/1000 sep=},
	]
	\addplot[const plot,blue] file {wykresy/zadanie4_DMC_Y.txt};
	\addplot[const plot,red, densely dashed] file {wykresy/zadanie4_DMC_Yzad.txt};
	\legend{$Y(k)$,$Y^{zad}(k)$}
	\end{axis}
	\end{tikzpicture}
	
	\begin{tikzpicture}
	\begin{axis}[
	width=4.667in,
	height=1.645in,
	xmin=0,xmax=2500,ymin=-1,ymax=3.5,
	xlabel={$k$},
	ylabel={$U$},
	legend pos=south east,
	y tick label style={/pgf/number format/1000 sep=},
	]
	\addplot[const plot,blue] file {wykresy/zadanie4_DMC_U.txt};
	\end{axis}
	\end{tikzpicture}

	\caption{Przebiegi sygna��w wej�ciowych i wyj�ciowych procesu}
	\label{dmcinz}
\end{figure}
\FloatBarrier